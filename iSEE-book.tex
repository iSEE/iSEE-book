\documentclass[]{book}
\usepackage{lmodern}
\usepackage{amssymb,amsmath}
\usepackage{ifxetex,ifluatex}
\usepackage{fixltx2e} % provides \textsubscript
\ifnum 0\ifxetex 1\fi\ifluatex 1\fi=0 % if pdftex
  \usepackage[T1]{fontenc}
  \usepackage[utf8]{inputenc}
\else % if luatex or xelatex
  \ifxetex
    \usepackage{mathspec}
  \else
    \usepackage{fontspec}
  \fi
  \defaultfontfeatures{Ligatures=TeX,Scale=MatchLowercase}
\fi
% use upquote if available, for straight quotes in verbatim environments
\IfFileExists{upquote.sty}{\usepackage{upquote}}{}
% use microtype if available
\IfFileExists{microtype.sty}{%
\usepackage[]{microtype}
\UseMicrotypeSet[protrusion]{basicmath} % disable protrusion for tt fonts
}{}
\PassOptionsToPackage{hyphens}{url} % url is loaded by hyperref
\usepackage[unicode=true]{hyperref}
\hypersetup{
            pdftitle={Extending iSEE},
            pdfauthor={Kevin Rue-Albrecht, Federico Marini, Charlotte Soneson, and Aaron Lun},
            pdfborder={0 0 0},
            breaklinks=true}
\urlstyle{same}  % don't use monospace font for urls
\usepackage{natbib}
\bibliographystyle{apalike}
\usepackage{color}
\usepackage{fancyvrb}
\newcommand{\VerbBar}{|}
\newcommand{\VERB}{\Verb[commandchars=\\\{\}]}
\DefineVerbatimEnvironment{Highlighting}{Verbatim}{commandchars=\\\{\}}
% Add ',fontsize=\small' for more characters per line
\usepackage{framed}
\definecolor{shadecolor}{RGB}{248,248,248}
\newenvironment{Shaded}{\begin{snugshade}}{\end{snugshade}}
\newcommand{\KeywordTok}[1]{\textcolor[rgb]{0.13,0.29,0.53}{\textbf{#1}}}
\newcommand{\DataTypeTok}[1]{\textcolor[rgb]{0.13,0.29,0.53}{#1}}
\newcommand{\DecValTok}[1]{\textcolor[rgb]{0.00,0.00,0.81}{#1}}
\newcommand{\BaseNTok}[1]{\textcolor[rgb]{0.00,0.00,0.81}{#1}}
\newcommand{\FloatTok}[1]{\textcolor[rgb]{0.00,0.00,0.81}{#1}}
\newcommand{\ConstantTok}[1]{\textcolor[rgb]{0.00,0.00,0.00}{#1}}
\newcommand{\CharTok}[1]{\textcolor[rgb]{0.31,0.60,0.02}{#1}}
\newcommand{\SpecialCharTok}[1]{\textcolor[rgb]{0.00,0.00,0.00}{#1}}
\newcommand{\StringTok}[1]{\textcolor[rgb]{0.31,0.60,0.02}{#1}}
\newcommand{\VerbatimStringTok}[1]{\textcolor[rgb]{0.31,0.60,0.02}{#1}}
\newcommand{\SpecialStringTok}[1]{\textcolor[rgb]{0.31,0.60,0.02}{#1}}
\newcommand{\ImportTok}[1]{#1}
\newcommand{\CommentTok}[1]{\textcolor[rgb]{0.56,0.35,0.01}{\textit{#1}}}
\newcommand{\DocumentationTok}[1]{\textcolor[rgb]{0.56,0.35,0.01}{\textbf{\textit{#1}}}}
\newcommand{\AnnotationTok}[1]{\textcolor[rgb]{0.56,0.35,0.01}{\textbf{\textit{#1}}}}
\newcommand{\CommentVarTok}[1]{\textcolor[rgb]{0.56,0.35,0.01}{\textbf{\textit{#1}}}}
\newcommand{\OtherTok}[1]{\textcolor[rgb]{0.56,0.35,0.01}{#1}}
\newcommand{\FunctionTok}[1]{\textcolor[rgb]{0.00,0.00,0.00}{#1}}
\newcommand{\VariableTok}[1]{\textcolor[rgb]{0.00,0.00,0.00}{#1}}
\newcommand{\ControlFlowTok}[1]{\textcolor[rgb]{0.13,0.29,0.53}{\textbf{#1}}}
\newcommand{\OperatorTok}[1]{\textcolor[rgb]{0.81,0.36,0.00}{\textbf{#1}}}
\newcommand{\BuiltInTok}[1]{#1}
\newcommand{\ExtensionTok}[1]{#1}
\newcommand{\PreprocessorTok}[1]{\textcolor[rgb]{0.56,0.35,0.01}{\textit{#1}}}
\newcommand{\AttributeTok}[1]{\textcolor[rgb]{0.77,0.63,0.00}{#1}}
\newcommand{\RegionMarkerTok}[1]{#1}
\newcommand{\InformationTok}[1]{\textcolor[rgb]{0.56,0.35,0.01}{\textbf{\textit{#1}}}}
\newcommand{\WarningTok}[1]{\textcolor[rgb]{0.56,0.35,0.01}{\textbf{\textit{#1}}}}
\newcommand{\AlertTok}[1]{\textcolor[rgb]{0.94,0.16,0.16}{#1}}
\newcommand{\ErrorTok}[1]{\textcolor[rgb]{0.64,0.00,0.00}{\textbf{#1}}}
\newcommand{\NormalTok}[1]{#1}
\usepackage{longtable,booktabs}
% Fix footnotes in tables (requires footnote package)
\IfFileExists{footnote.sty}{\usepackage{footnote}\makesavenoteenv{long table}}{}
\usepackage{graphicx,grffile}
\makeatletter
\def\maxwidth{\ifdim\Gin@nat@width>\linewidth\linewidth\else\Gin@nat@width\fi}
\def\maxheight{\ifdim\Gin@nat@height>\textheight\textheight\else\Gin@nat@height\fi}
\makeatother
% Scale images if necessary, so that they will not overflow the page
% margins by default, and it is still possible to overwrite the defaults
% using explicit options in \includegraphics[width, height, ...]{}
\setkeys{Gin}{width=\maxwidth,height=\maxheight,keepaspectratio}
\IfFileExists{parskip.sty}{%
\usepackage{parskip}
}{% else
\setlength{\parindent}{0pt}
\setlength{\parskip}{6pt plus 2pt minus 1pt}
}
\setlength{\emergencystretch}{3em}  % prevent overfull lines
\providecommand{\tightlist}{%
  \setlength{\itemsep}{0pt}\setlength{\parskip}{0pt}}
\setcounter{secnumdepth}{5}
% Redefines (sub)paragraphs to behave more like sections
\ifx\paragraph\undefined\else
\let\oldparagraph\paragraph
\renewcommand{\paragraph}[1]{\oldparagraph{#1}\mbox{}}
\fi
\ifx\subparagraph\undefined\else
\let\oldsubparagraph\subparagraph
\renewcommand{\subparagraph}[1]{\oldsubparagraph{#1}\mbox{}}
\fi

% set default figure placement to htbp
\makeatletter
\def\fps@figure{htbp}
\makeatother

\usepackage{booktabs}

\title{Extending \emph{iSEE}}
\author{Kevin Rue-Albrecht, Federico Marini, Charlotte Soneson, and Aaron Lun}
\date{2020-02-23}

\begin{document}
\maketitle

{
\setcounter{tocdepth}{1}
\tableofcontents
}
\chapter*{Preface}\label{preface}
\addcontentsline{toc}{chapter}{Preface}

The \href{https://bioconductor.org/}{Bioconductor} package
\emph{\href{https://bioconductor.org/packages/3.11/iSEE}{iSEE}} provides
functions for creating an interactive graphical user interface (GUI)
using the \href{https://rstudio.com/}{RStudio}
\emph{\href{https://CRAN.R-project.org/package=Shiny}{Shiny}} package
for exploring data stored in
\emph{\href{https://bioconductor.org/packages/3.11/SummarizedExperiment}{SummarizedExperiment}}
objects, including row- and column-level metadata \citep{rue2018isee}.
In this book we describe key concepts and case studies to create
web-applications that leverage builtin panels and develop new ones. We
also present case studies to illustrate the development of custom
panels.

\hypertarget{panels}{\chapter{Panel classes}\label{panels}}

\section{Overview}\label{overview}

The types of builtin panels available to compose an
\emph{\href{https://bioconductor.org/packages/3.11/iSEE}{iSEE}} app are
defined as a hierarchy of S4 classes.

\begin{itemize}
\tightlist
\item
  \texttt{Panel}

  \begin{itemize}
  \tightlist
  \item
    \texttt{DotPlot}

    \begin{itemize}
    \tightlist
    \item
      \texttt{ColumnDotPlot}

      \begin{itemize}
      \tightlist
      \item
        \texttt{ReducedDimensionPlot}
      \item
        \texttt{ColumnDataPlot}
      \item
        \texttt{FeatureAssayPlot}
      \end{itemize}
    \item
      \texttt{RowDotPlot}

      \begin{itemize}
      \tightlist
      \item
        \texttt{RowDataPlot}
      \item
        \texttt{SampleAssayPlot}
      \end{itemize}
    \end{itemize}
  \item
    \texttt{Table}

    \begin{itemize}
    \tightlist
    \item
      \texttt{RowTable}

      \begin{itemize}
      \tightlist
      \item
        \texttt{RowDataTable}
      \end{itemize}
    \item
      \texttt{ColumnTable}

      \begin{itemize}
      \tightlist
      \item
        \texttt{ColDataTable}
      \end{itemize}
    \end{itemize}
  \item
    \texttt{ComplexHeatmapPlot}
  \end{itemize}
\end{itemize}

Some of those classes are ``virtual'' (indicated by ), meaning that they
cannot be directly instanciated as panels in the GUI. Instead, virtual
panel classes define families of panels that share groups of properties.
As such, virtual classes are meant to be used as the parent of concrete
classes that share the associated properties.

In contrast, concrete classes must define fully-functional panels that
can be embedded in a GUI, interact with other panels, receive and
process data, and generate an output such as a plot or a table,
accompanied by the associated R code to display in the code tracker for
reproducibility.

\section{The Panel class}\label{the-panel-class}

The top-most class is called \texttt{Panel}. It is a virtual class that
defines the core properties common to any panel - existing or future -
that may be displayed in the interface.

Refer to \texttt{help("Panel-class",\ "iSEE")} for more information
about the slots and methods provided by this class.

\section{The DotPlot and Table panel
families}\label{the-dotplot-and-table-panel-families}

The virtual class \texttt{Panel} is directly derived into two major
virtual sub-classes:

\begin{itemize}
\tightlist
\item
  \texttt{DotPlot}
\item
  \texttt{Table}
\end{itemize}

Those classes introduce properties that are specific to distinct subsets
of panel types.

The class \texttt{DotPlot} introduces parameters specific to panels
where the output is a \texttt{ggplot} object and each row in the
data-frame is represented as a point in a plot. Refer to
\texttt{help("DotPlot-class",\ "iSEE")} for more information.

The class \texttt{Table} introduces parameters specific to panels where
the main output is a data-frame directly displayed as a table in the
GUI. Refer to \texttt{help("Table-class",\ "iSEE")} for more
information.

As a special case, the class \texttt{ComplexHeatmapPlot} defines a
concrete panel class that directly extends the class \texttt{Panel}, as
it introduces a set of parameters distinct from both the
\texttt{DotPlot} and \texttt{Table} panel families. This panel class is
described in further details in the section
\protect\hyperlink{complexheatmapplot-class}{The ComplexHeatmapPlot
panel class}.

\section{The ColumnDotPlot and RowDotPlot panel
families}\label{the-columndotplot-and-rowdotplot-panel-families}

\section{Built-in ColumnDotPlot panel
classes}\label{built-in-columndotplot-panel-classes}

\section{Built-in RowDotPlot panel
classes}\label{built-in-rowdotplot-panel-classes}

\section{The ColumnTable and RowTable panel
families}\label{the-columntable-and-rowtable-panel-families}

\section{Built-in ColumnTable panel
classes}\label{built-in-columntable-panel-classes}

\section{Built-in RowTable panel
classes}\label{built-in-rowtable-panel-classes}

\hypertarget{complexheatmapplot-class}{\section{The ComplexHeatmapPlot
panel class}\label{complexheatmapplot-class}}

This type of panel introduces parameters specific to panels where the
output is a \texttt{Heatmap} object from the
\emph{\href{https://bioconductor.org/packages/3.11/ComplexHeatmap}{ComplexHeatmap}}
package. In this panel, each row represents a feature and each column
represents a sample in the \texttt{se} object.

\chapter{The panel API}\label{api}

\section{.cacheCommonInfo}\label{cachecommoninfo}

Each individual panel type (e.g., \texttt{ReducedDimensionPlot}) and
family of panel types (e.g., \texttt{ColumnDotPlot}) defines a signature
for the method \texttt{.cacheCommonInfo()}.

This function is called for each panel instance in memory when the app
is initialized. It allows the app to efficiently compute a single time
common information that depends only on the input \texttt{se} object,
and that may be frequently reused during the runtime of an app.

Following the hierarchy of panel types, each call to the signature takes
a panel instance \texttt{x} and the \texttt{se} object, and caches
common information relevant to any instance of that panel type in the
\texttt{se} object itself, before calling \texttt{callNextMethod()} to
invoke the next parent signature.

The top-most signature - for the \texttt{Panel} class - returns the
\texttt{se} object that contains all the cached information.

Note that this function only populates the cache for the first panel of
each type; it is a no-op if the common cache has already been
initialized.

\section{.refineParameters}\label{refineparameters}

Each individual panel type (e.g., \texttt{ReducedDimensionPlot}) and
family of panel types (e.g., \texttt{ColumnDotPlot}) defines a signature
for the method \texttt{.refineParameters()}.

This function is called for each panel instance in memory when the app
is initialized, and also for each new panel added to the GUI at runtime.
It inspects the parameters of the given panel instance, and replaces any
invalid parameter with a sensible value for a given \texttt{se} object.

Following the hierarchy of panel types, each call to the signature takes
a panel instance \texttt{x} and the \texttt{se} object, and first calls
\texttt{callNextMethod()} to invoke the next parent signature, to refine
generic parameters before processing specific ones.

The called signature ultimately returns the updated panel instance
\texttt{x}, or \texttt{NULL} if the panel instance is not available for
this app.

\section{.generateDotPlot}\label{generatedotplot}

Each panel type that derives from the virtual class \texttt{DotPlot}
must define - or inherit - a signature for the method
\texttt{.generateDotPlot()}.

This function is called within \texttt{.renderOutput()}, which is
triggered by app observers when the value of the input widgets are
changed by users, or when a new panel is added to the GUI.

The method \texttt{.generateDotPlot()} has access to the parameters for
a given panel instance, and uses information available in the panel
evaluation environment to generate and evaluate the plotting commands
that ultimately produce the \texttt{ggplot} object to display in the
panel.

Refer to the \textbf{``Generating the ggplot object''} section of
\texttt{help(".generateDotPlot",\ "iSEE")} for more information.

\chapter{The app server}\label{server}

\hypertarget{robjects}{\section{Reactive objects}\label{robjects}}

\section{Persistent (non-reactive)
objects}\label{persistent-non-reactive-objects}

\hypertarget{memory}{\section{The app memory}\label{memory}}

The app \texttt{memory} is a list of instances created from available
panel classes and currently visible in the GUI. The order of panel
instances in \texttt{memory} directly reflects their order in the GUI.

\section{Initialization of the app
server}\label{initialization-of-the-app-server}

The app server is initialized as soon as a valid \texttt{se} object is
provided. This can be either in the call to \texttt{iSEE(se)} or using
other mechanisms of data upload at runtime within the app (e.g.,
\texttt{fileInput} UI widgets when \texttt{iSEE()} is called without
providing the \texttt{se} argument).

The internal function \texttt{iSEE:::.initialize\_server()} takes the
\texttt{se} object and the list holding reactive values used to trigger
re-rendering of the GUI, as described in the section
\protect\hyperlink{robjects}{Reactive objects}.

The very first step invokes the internal function
\texttt{checkColormapCompatibility()}. This function takes the
\texttt{se} object and the optional \texttt{colormap} argument provided
to \texttt{iSEE()}, and carries out a number of compatibility checks
between the two objects. The function collects a character vector of
incompatibility issues that are displayed - if any - as warning
notifications in GUI when the app is launched.

Next, the internal function \texttt{iSEE:::.prepare\_SE()} calls the
method \texttt{.cacheCommonInfo()} on each type of panel present
provided to \texttt{iSEE(initial)} and \texttt{iSEE(extra)}, to
precompute and cache information relevant to all the types of panels
that will be available in that app instance.

Shortly after, the internal function
\texttt{iSEE:::.setup\_initial\_state()} calls the method
\texttt{.refineParameters} on each panel instance provided to
\texttt{iSEE(initial)}, to ensure that all the panels present in the GUI
when the app is launched are initialized with valid parameters; any
invalid parameter is replaced with sensible values for the given
\texttt{se} object.

The internal function \texttt{iSEE:::.create\_persistent\_objects()}
executes the initialization of persistent (non-reactive) objects:

\begin{itemize}
\tightlist
\item
  the app \texttt{memory} (see the section
  \protect\hyperlink{memory}{The app memory})
\item
  the app \texttt{reservoir}, which stores one instance of panel type
  available for this app instance
\item
  the app \texttt{counter} is used to track the number of panels
  previously created for each type, and to assign an increasing
  identifier to new panel instances
\item
  the app \texttt{commands} stores the list of code chunks to display in
  the code tracker, to reproduce each panel output
\item
  the app \texttt{contents} stores the list of data point coordinates
  selectable in each panel instance\footnote{Data points downsampled for
    rendering speed performance remain selectable, even though they are
    not visible in the plot.} 
\item
  the identifier of the panel under the control of speech recognition
\end{itemize}

\hypertarget{developing}{\chapter{Developing new
panels}\label{developing}}

First, we need to load the
\emph{\href{https://bioconductor.org/packages/3.11/iSEE}{iSEE}} package
for this chapter. This action imports all the builtin panel class
definitions, including the virtual class \texttt{Panel} that is the base
class for any
\emph{\href{https://bioconductor.org/packages/3.11/iSEE}{iSEE}} panel
class.

\begin{Shaded}
\begin{Highlighting}[]
\KeywordTok{library}\NormalTok{(iSEE)}
\end{Highlighting}
\end{Shaded}

We also set up an example using our favorite dataset, creating a
\texttt{SingleCellExperiment} object with some precomputed
dimensionality reduction results.

\begin{Shaded}
\begin{Highlighting}[]
\KeywordTok{library}\NormalTok{(scRNAseq)}
\NormalTok{sce <-}\StringTok{ }\KeywordTok{ReprocessedAllenData}\NormalTok{(}\DataTypeTok{assays=}\StringTok{"tophat_counts"}\NormalTok{)}

\KeywordTok{library}\NormalTok{(scater)}
\NormalTok{sce <-}\StringTok{ }\KeywordTok{logNormCounts}\NormalTok{(sce, }\DataTypeTok{exprs_values=}\StringTok{"tophat_counts"}\NormalTok{)}
\NormalTok{sce <-}\StringTok{ }\KeywordTok{runPCA}\NormalTok{(sce, }\DataTypeTok{ncomponents=}\DecValTok{4}\NormalTok{)}
\NormalTok{sce <-}\StringTok{ }\KeywordTok{runTSNE}\NormalTok{(sce)}
\end{Highlighting}
\end{Shaded}

\section{Create a new S4 class}\label{create-a-new-s4-class}

In the chapter \protect\hyperlink{panels}{Panel classes}, we saw how
each type of panel is defined as an S4 class, organised in a hierarchy
that allows new panel classes to inherit sets of the properties and
functionality from parent classes.

Then, developing a new panel type starts with the creation of a new
class that inherits from the \texttt{Panel} class.

While it is possible to create a new panel class that directly inherits
from the top-most virtual \texttt{Panel} class, this is the most
advanced use case that we will describe in later chapters.

Instead, new concrete panels classes can be rapidly derived from other
concrete parent panel classes, using the inheritance relationships
between classes to reuse properties and functionality defined in all of
the parent classes.

The choice of a parent class depends on the properties that we want that
new panel class to start with. For instance, to create a panel that
inherits all the functionality of the \texttt{ReducedDimensionPlot}
panel type, we simply define a new class that extends that class. For
example in this chapter, we call that new class \texttt{RedDimHexPlot}.

\begin{Shaded}
\begin{Highlighting}[]
\KeywordTok{setClass}\NormalTok{(}\StringTok{"RedDimHexPlot"}\NormalTok{, }\DataTypeTok{contains=}\StringTok{"ReducedDimensionPlot"}\NormalTok{)}
\end{Highlighting}
\end{Shaded}

\section{Add a constructor function}\label{add-a-constructor-function}

At this point, it is already possible to create instances of the new
panel class. To facilitate this, new panels should provide a constructor
function - best practice is to name it identically to the class - to
accept arbitrary arguments controlling the initialization of new panel
instances created by the function \texttt{new()}.

Here, we define a simple constructor function that passes all incoming
arguments \emph{as is} to \texttt{new()}.

\begin{Shaded}
\begin{Highlighting}[]
\NormalTok{RedDimHexPlot <-}\StringTok{ }\ControlFlowTok{function}\NormalTok{(...) \{}
    \KeywordTok{new}\NormalTok{(}\StringTok{"RedDimHexPlot"}\NormalTok{, ...)}
\NormalTok{\}}
\end{Highlighting}
\end{Shaded}

At this point, we can already use instances of this new panel class in
\emph{\href{https://bioconductor.org/packages/3.11/iSEE}{iSEE}} apps.

\begin{Shaded}
\begin{Highlighting}[]
\NormalTok{RedDimHexPlot1 <-}\StringTok{ }\KeywordTok{RedDimHexPlot}\NormalTok{()}
\NormalTok{initial <-}\StringTok{ }\KeywordTok{list}\NormalTok{(RedDimHexPlot1)}
\NormalTok{app <-}\StringTok{ }\KeywordTok{iSEE}\NormalTok{(sce, }\DataTypeTok{initial =}\NormalTok{ initial)}
\end{Highlighting}
\end{Shaded}

However, that would not be very exciting as instances of this new panel
class would behave exactly like the those of the parent
\texttt{ReducedDimensionPlot} class itself. To illustrate this, the
following code chunk initializes an app displaying an instance of the
new \texttt{RedDimHexPlot} and its parent \texttt{ReducedDimensionPlot}
side by side.

\begin{Shaded}
\begin{Highlighting}[]
\NormalTok{RedDimHexPlot1 <-}\StringTok{ }\KeywordTok{RedDimHexPlot}\NormalTok{()}
\NormalTok{ReducedDimensionPlot1 <-}\StringTok{ }\KeywordTok{ReducedDimensionPlot}\NormalTok{()}
\NormalTok{initial <-}\StringTok{ }\KeywordTok{list}\NormalTok{(RedDimHexPlot1, ReducedDimensionPlot1)}
\NormalTok{app <-}\StringTok{ }\KeywordTok{iSEE}\NormalTok{(sce, }\DataTypeTok{initial =}\NormalTok{ initial)}
\end{Highlighting}
\end{Shaded}

\section{Set the panel name in the
GUI}\label{set-the-panel-name-in-the-gui}

The panel class that we created so far also inherited the name of the
parent panel class. In other words, instances of both classes are
entirely indistinguishable from each other in the GUI.

The name of each panel displayed in the GUI is defined by the method
\texttt{.fullName()}. To clearly distinguish the new panel class in the
GUI, we overwrite this method to display a name different from the
parent class.

\begin{Shaded}
\begin{Highlighting}[]
\KeywordTok{setMethod}\NormalTok{(}\StringTok{".fullName"}\NormalTok{, }\StringTok{"RedDimHexPlot"}\NormalTok{, }\ControlFlowTok{function}\NormalTok{(x) }\StringTok{"Reduced dimension hexagonal plot"}\NormalTok{)}
\end{Highlighting}
\end{Shaded}

With that, launching \texttt{app} again now highlights how panels of the
new class now display a different title from the parent class.

\section{Define the commands generating a plot
output}\label{define-the-commands-generating-a-plot-output}

Importantly, the API separates the generation of commands processing
data from \texttt{sce} into a data-frame, from the generation of
commands producing a \texttt{ggplot} object using the processed
data-frame. If a new panel class derived from \texttt{DotPlot} is meant
to process data in the same way as its parent panel, only to display in
a different way, it is then possible to overwrite only the method
\texttt{.generateDotPlot()}. Meanwhile, the data preprocessing will be
implicitly handled by the other API methods inherited from the parent
class.

Importantly, the method \texttt{.generateDotPlot()} requires two key
arguments: \texttt{labels} provides the plot labels for each of the
aesthetics in the plot data, and \texttt{envir} provides the environment
in which the plotting commands are to be evaluated to produce the
\texttt{ggplot} object.

In particular, the contract offered by iSEE to panel developers promises
the presence of certain variables in \texttt{envir}, that
\texttt{.generateDotPlot()} can rely on. Using those environment
variables, \texttt{.generateDotPlot()} can make decisions altering the
plotting commands and the resulting \texttt{ggplot} object. For
instance, the most important environment variable is \texttt{plot.data},
the data-frame that contains one row per data point to display in
\texttt{DotPlot} panels.

Readers should refer to the \textbf{``Generating the ggplot object''}
section of \texttt{help(".generateDotPlot",\ "iSEE")} for more
information.

As an example, we overwrite the method \texttt{.generateDotPlot()} for
the new class \texttt{RedDimHexPlot} to simply show the number of data
points in the plotting area as a heatmap dividing the plane into regular
hexagons. Notably, the contract described above guarantees that the
function can immediately rely on the \texttt{plot.data} data-frame that
is computed by methods defined for the parent class
\texttt{ReducedDimensionPlot}. We also use the precomputed aesthetic
\texttt{labels} associated with each column of \texttt{plot.data}, while
setting a fixed label \texttt{"Count"} for the \texttt{fill} aesthetic
associated with the count of observation in each hexagonal bin.

\begin{Shaded}
\begin{Highlighting}[]
\KeywordTok{setMethod}\NormalTok{(}\StringTok{".generateDotPlot"}\NormalTok{, }\StringTok{"RedDimHexPlot"}\NormalTok{, }\ControlFlowTok{function}\NormalTok{(x, labels, envir) \{}
    \KeywordTok{stopifnot}\NormalTok{(}\KeywordTok{require}\NormalTok{(ggplot2))}
    
\NormalTok{    plot_cmds <-}\StringTok{ }\KeywordTok{list}\NormalTok{()}
\NormalTok{    plot_cmds[[}\StringTok{"ggplot"}\NormalTok{]] <-}\StringTok{ "ggplot() +"}
    
    \CommentTok{# Adding hexbins to the plot.}
\NormalTok{    plot_cmds[[}\StringTok{"hex"}\NormalTok{]] <-}\StringTok{ "geom_hex(aes(X, Y), plot.data) +"}
\NormalTok{    plot_cmds[[}\StringTok{"labs"}\NormalTok{]] <-}\StringTok{ "labs(fill='Count') +"}
\NormalTok{    plot_cmds[[}\StringTok{"labs"}\NormalTok{]] <-}\StringTok{ }\KeywordTok{sprintf}\NormalTok{(}
        \StringTok{"labs(x='%s', y='%s', title='%s', fill='%s') +"}\NormalTok{,}
\NormalTok{        labels}\OperatorTok{$}\NormalTok{X, labels}\OperatorTok{$}\NormalTok{Y, labels}\OperatorTok{$}\NormalTok{title, }\StringTok{"Count"}
\NormalTok{        )}
\NormalTok{    plot_cmds[[}\StringTok{"theme_base"}\NormalTok{]] <-}\StringTok{ "theme_bw() +"}
\NormalTok{    plot_cmds[[}\StringTok{"theme_legend"}\NormalTok{]] <-}\StringTok{ "theme(legend.position = 'bottom')"}
    
\NormalTok{    gg_plot <-}\StringTok{ }\KeywordTok{eval}\NormalTok{(}\KeywordTok{parse}\NormalTok{(}\DataTypeTok{text=}\NormalTok{plot_cmds), envir)}
    
    \KeywordTok{list}\NormalTok{(}\DataTypeTok{plot=}\NormalTok{gg_plot, }\DataTypeTok{commands=}\NormalTok{plot_cmds)}
\NormalTok{\})}
\end{Highlighting}
\end{Shaded}

Running \texttt{app} again highlights how the \texttt{RedDimHexPlot}
panel fills each hexagonal bin with a color indicating the number of
data points present in the corresponding area in the
\texttt{ReducedDimensionPlot} panel.

\chapter{Dynamic reduced dimensions}\label{dynamic-reduced-dimensions}

\section{Overview}\label{overview-1}

In this case study, we will create a custom panel class to regenerate
sample-level PCA coordinates using only a subset of points transmitted
as a multiple column selection from another panel. We call this a
\textbf{dynamic reduced dimension plot}, as it is dynamically
recomputing the dimensionality reduction results rather than using
pre-computed values in the \texttt{reducedDims()} slot of a
\texttt{SingleCellExperiment} object.

\section{Class basics}\label{class-basics}

First, we define the basics of our new \texttt{Panel} class. As our new
class will be showing each sample as a point, we inherit from the
\texttt{ColumnDotPlot} virtual class. This automatically gives us access
to all the functionality promised in the contract, including interface
elements and observers to handle multiple selections and respond to
aesthetic parameters.

We add a slot specifying the type of dimensionality reduction result and
the number of highly variable genes to use. Any new slots should also
come with validity methods, as shown below.

\begin{Shaded}
\begin{Highlighting}[]
\KeywordTok{library}\NormalTok{(S4Vectors)}
\KeywordTok{setValidity2}\NormalTok{(}\StringTok{"DynReducedDimensionPlot"}\NormalTok{, }\ControlFlowTok{function}\NormalTok{(object) \{}
\NormalTok{    msg <-}\StringTok{ }\KeywordTok{character}\NormalTok{(}\DecValTok{0}\NormalTok{)}

    \ControlFlowTok{if}\NormalTok{ (}\KeywordTok{length}\NormalTok{(n <-}\StringTok{ }\NormalTok{object[[}\StringTok{"NGenes"}\NormalTok{]])}\OperatorTok{!=}\NormalTok{1L }\OperatorTok{||}\StringTok{ }\NormalTok{n }\OperatorTok{<}\StringTok{ }\NormalTok{1L) \{}
\NormalTok{        msg <-}\StringTok{ }\KeywordTok{c}\NormalTok{(msg, }\StringTok{"'NGenes' must be a positive integer scalar"}\NormalTok{) }
\NormalTok{    \}}
    \ControlFlowTok{if}\NormalTok{ (}\OperatorTok{!}\KeywordTok{isSingleString}\NormalTok{(val <-}\StringTok{ }\NormalTok{object[[}\StringTok{"Type"}\NormalTok{]]) }\OperatorTok{||}\StringTok{ }
\StringTok{        }\OperatorTok{!}\NormalTok{val }\OperatorTok\StringTok{ }\KeywordTok{c}\NormalTok{(}\StringTok{"PCA"}\NormalTok{, }\StringTok{"TSNE"}\NormalTok{, }\StringTok{"UMAP"}\NormalTok{)) }
\NormalTok{    \{}
\NormalTok{        msg <-}\StringTok{ }\KeywordTok{c}\NormalTok{(msg, }\StringTok{"'Type' must be one of 'TSNE', 'PCA' or 'UMAP'"}\NormalTok{)}
\NormalTok{    \}}

    \ControlFlowTok{if}\NormalTok{ (}\KeywordTok{length}\NormalTok{(msg)) \{}
        \KeywordTok{return}\NormalTok{(msg)}
\NormalTok{    \}}
    \OtherTok{TRUE}
\NormalTok{\})}
\end{Highlighting}
\end{Shaded}

It is also worthwhile specializing the \texttt{initialize()} method to
provide a default for new parameters:

\begin{Shaded}
\begin{Highlighting}[]
\KeywordTok{setMethod}\NormalTok{(}\StringTok{"initialize"}\NormalTok{, }\StringTok{"DynReducedDimensionPlot"}\NormalTok{, }
    \ControlFlowTok{function}\NormalTok{(.Object, }\DataTypeTok{Type=}\StringTok{"PCA"}\NormalTok{, }\DataTypeTok{NGenes=}\NormalTok{1000L, ...) }
\NormalTok{\{}
    \KeywordTok{callNextMethod}\NormalTok{(.Object, }\DataTypeTok{Type=}\NormalTok{Type, }\DataTypeTok{NGenes=}\NormalTok{NGenes, ...)}
\NormalTok{\})}
\end{Highlighting}
\end{Shaded}

\section{Setting up the interface}\label{setting-up-the-interface}

The most basic requirement is to define some methods that describe our
new panel in the \texttt{iSEE()} interface. This includes defining the
full name and desired default color for display purposes:

\begin{Shaded}
\begin{Highlighting}[]
\KeywordTok{setMethod}\NormalTok{(}\StringTok{".fullName"}\NormalTok{, }\StringTok{"DynReducedDimensionPlot"}\NormalTok{, }\ControlFlowTok{function}\NormalTok{(x) }\StringTok{"Dynamic reduced dimension plot"}\NormalTok{)}

\KeywordTok{setMethod}\NormalTok{(}\StringTok{".panelColor"}\NormalTok{, }\StringTok{"DynReducedDimensionPlot"}\NormalTok{, }\ControlFlowTok{function}\NormalTok{(x) }\StringTok{"#0F0F0F"}\NormalTok{)}
\end{Highlighting}
\end{Shaded}

We also add interface elements to change the result type and the number
of genes. This is most easily done by specializing the
\texttt{.defineDataInterface} method:

\begin{Shaded}
\begin{Highlighting}[]
\KeywordTok{library}\NormalTok{(shiny)}
\KeywordTok{setMethod}\NormalTok{(}\StringTok{".defineDataInterface"}\NormalTok{, }\StringTok{"DynReducedDimensionPlot"}\NormalTok{, }\ControlFlowTok{function}\NormalTok{(x, se, select_info) \{}
\NormalTok{    plot_name <-}\StringTok{ }\KeywordTok{.getEncodedName}\NormalTok{(x)}

    \KeywordTok{list}\NormalTok{(}
        \KeywordTok{selectInput}\NormalTok{(}\KeywordTok{paste0}\NormalTok{(plot_name, }\StringTok{"_Type"}\NormalTok{), }\DataTypeTok{label=}\StringTok{"Type:"}\NormalTok{,}
            \DataTypeTok{choices=}\KeywordTok{c}\NormalTok{(}\StringTok{"PCA"}\NormalTok{, }\StringTok{"TSNE"}\NormalTok{, }\StringTok{"UMAP"}\NormalTok{), }\DataTypeTok{selected=}\NormalTok{x[[}\StringTok{"Type"}\NormalTok{]]),}
        \KeywordTok{numericInput}\NormalTok{(}\KeywordTok{paste0}\NormalTok{(plot_name, }\StringTok{"_NGenes"}\NormalTok{), }\DataTypeTok{label=}\StringTok{"Number of HVGs:"}\NormalTok{,}
            \DataTypeTok{min=}\DecValTok{1}\NormalTok{, }\DataTypeTok{value=}\NormalTok{x[[}\StringTok{"NGenes"}\NormalTok{]])}
\NormalTok{    )}
\NormalTok{\})}
\end{Highlighting}
\end{Shaded}

We call \texttt{.getEncodedName()} to obtain a unique name for the
current instance of our panel, e.g., \texttt{DynReducedDimensionPlot1}.
We then \texttt{paste0} the name of our panel to the name of any
parameter to ensure that the ID is unique to this instance of our panel;
otherwise, multiple \texttt{DynReducedDimensionPlot}s would override
each other. One can imagine this as a poor man's Shiny module.

\section{Creating the observers}\label{creating-the-observers}

We specialize \texttt{.createObservers} to define some observers to
respond to changes in our new interface elements. Note the use of
\texttt{callNextMethod()} to ensure that observers of the parent class
are also created; this automatically ensures that we can respond to
changes in parameters provided by \texttt{ColumnDotPlot}.

\begin{Shaded}
\begin{Highlighting}[]
\KeywordTok{setMethod}\NormalTok{(}\StringTok{".createObservers"}\NormalTok{, }\StringTok{"DynReducedDimensionPlot"}\NormalTok{, }
    \ControlFlowTok{function}\NormalTok{(x, se, input, session, pObjects, rObjects) }
\NormalTok{\{}
    \KeywordTok{callNextMethod}\NormalTok{()}

\NormalTok{    plot_name <-}\StringTok{ }\KeywordTok{.getEncodedName}\NormalTok{(x)}

    \KeywordTok{.createProtectedParameterObservers}\NormalTok{(plot_name,}
        \DataTypeTok{fields=}\KeywordTok{c}\NormalTok{(}\StringTok{"Type"}\NormalTok{, }\StringTok{"NGenes"}\NormalTok{),}
        \DataTypeTok{input=}\NormalTok{input, }\DataTypeTok{pObjects=}\NormalTok{pObjects, }\DataTypeTok{rObjects=}\NormalTok{rObjects)}
\NormalTok{\})}
\end{Highlighting}
\end{Shaded}

Both the \texttt{NGenes} and \texttt{Type} parameters are what we
consider to be ``protected'' parameters, as changing them will alter the
nature of the displayed plot. We use the
\texttt{.createProtectedParameterObservers()} utility to set up
observers for both parameters, which will instruct \texttt{iSEE()} to
destroy existing brushes and lassos when these parameters are changed.
The idea here is that brushes/lassos made on the previous plot do not
make sense when the coordinates are recomputed.

\section{Making the plot}\label{making-the-plot}

When working with a \texttt{ColumnDotPlot} subclass, the easiest way to
change plotting content to override the \texttt{.generateDotPlotData}
method. This should add a \texttt{plot.data} variable to the
\texttt{envir} environment that has columns \texttt{X} and \texttt{Y}
and contains one row per column of the original
\texttt{SummarizedExperiment}. It should also return a character vector
of R commands describing how that \texttt{plot.data} object was
constructed. The easiest way to do this is to create a character vector
of commands and call \texttt{eval(parse(text=...),\ envir=envir)} to
evaluate them within \texttt{envir}.

\begin{Shaded}
\begin{Highlighting}[]
\KeywordTok{setMethod}\NormalTok{(}\StringTok{".generateDotPlotData"}\NormalTok{, }\StringTok{"DynReducedDimensionPlot"}\NormalTok{, }\ControlFlowTok{function}\NormalTok{(x, envir) \{}
\NormalTok{    commands <-}\StringTok{ }\KeywordTok{character}\NormalTok{(}\DecValTok{0}\NormalTok{)}

    \ControlFlowTok{if}\NormalTok{ (}\OperatorTok{!}\KeywordTok{exists}\NormalTok{(}\StringTok{"col_selected"}\NormalTok{, }\DataTypeTok{envir=}\NormalTok{envir, }\DataTypeTok{inherits=}\OtherTok{FALSE}\NormalTok{)) \{}
\NormalTok{        commands <-}\StringTok{ }\KeywordTok{c}\NormalTok{(commands, }
            \StringTok{"plot.data <- data.frame(X=numeric(0), Y=numeric(0));"}\NormalTok{)}
\NormalTok{    \} }\ControlFlowTok{else}\NormalTok{ \{}
\NormalTok{        commands <-}\StringTok{ }\KeywordTok{c}\NormalTok{(commands,}
            \StringTok{".chosen <- unique(unlist(col_selected));"}\NormalTok{,}
            \StringTok{"set.seed(100000)"}\NormalTok{, }\CommentTok{# to avoid problems with randomization.}
            \KeywordTok{sprintf}\NormalTok{(}\StringTok{".coords <- scater::calculate%s(se[,.chosen], ntop=%i, ncomponents=2);"}\NormalTok{,}
\NormalTok{                x[[}\StringTok{"Type"}\NormalTok{]], x[[}\StringTok{"NGenes"}\NormalTok{]]),}
            \StringTok{"plot.data <- data.frame(.coords, row.names=.chosen);"}\NormalTok{,}
            \StringTok{"colnames(plot.data) <- c('X', 'Y');"}
\NormalTok{        )}
\NormalTok{    \}}

\NormalTok{    commands <-}\StringTok{ }\KeywordTok{c}\NormalTok{(commands,}
        \StringTok{"plot.data <- plot.data[colnames(se),,drop=FALSE];"}\NormalTok{,}
        \StringTok{"rownames(plot.data) <- colnames(se);"}\NormalTok{)}

    \KeywordTok{eval}\NormalTok{(}\KeywordTok{parse}\NormalTok{(}\DataTypeTok{text=}\NormalTok{commands), }\DataTypeTok{envir=}\NormalTok{envir)}

    \KeywordTok{list}\NormalTok{(}\DataTypeTok{data_cmds=}\NormalTok{commands, }\DataTypeTok{plot_title=}\KeywordTok{sprintf}\NormalTok{(}\StringTok{"Dynamic %s plot"}\NormalTok{, x[[}\StringTok{"Type"}\NormalTok{]]), }
        \DataTypeTok{x_lab=}\KeywordTok{paste0}\NormalTok{(x[[}\StringTok{"Type"}\NormalTok{]], }\StringTok{"1"}\NormalTok{), }\DataTypeTok{y_lab=}\KeywordTok{paste0}\NormalTok{(x[[}\StringTok{"Type"}\NormalTok{]], }\StringTok{"2"}\NormalTok{))}
\NormalTok{\})}
\end{Highlighting}
\end{Shaded}

We use functions from the
\emph{\href{https://bioconductor.org/packages/3.11/scater}{scater}}
package to do the actual heavy lifting of calculating the dimensionality
reduction results. The \texttt{exists()} call will check whether any
column selection is being transmitted to this panel; if not, it will
just return a \texttt{plot.data} variable that contains all \texttt{NA}s
such that an empty plot is created. If \texttt{col\_selected} does
exist, it will contain a list of character vectors specifying the active
and saved multiple selections that are being transmitted. For this
particular example, we do not care about the distinction between
active/saved selections so we just take the union of all of them.

Of course, this is not quite the most efficient way to implement a
plotting panel that involves recomputation. A better approach would be
to cache the x/y coordinates and reuse them if only aesthetic parameters
have changed, thus avoiding an unnecessary delay from recomputation.
Doing so requires overriding \texttt{.renderOutput()} to take advantage
of the cached contents of the plot, so we will omit that here for
simplicity.

\section{Finishing touches}\label{finishing-touches}

For this particular panel class, an additional helpful feature is to
override \texttt{.multiSelectionInvalidated}. This indicates that any
brushes or lassos in our plot should be destroyed when we receive a new
column selection. Doing so is the only sensible course of action as the
reduced dimension coordinates for one set of samples have no obvious
relationship to the coordinates for another set of samples; having old
brushes or lassos hanging around would be of no benefit at best, and be
misleading at worst.

\begin{Shaded}
\begin{Highlighting}[]
\KeywordTok{setMethod}\NormalTok{(}\StringTok{".multiSelectionInvalidated"}\NormalTok{, }\StringTok{"DynReducedDimensionPlot"}\NormalTok{, }\ControlFlowTok{function}\NormalTok{(x) }\OtherTok{TRUE}\NormalTok{)}
\end{Highlighting}
\end{Shaded}

\section{In action}\label{in-action}

Let's put our new panel to the test. We use the \texttt{sce} object,
preprocessed in a \protect\hyperlink{developing}{previous chapter},
including some precomputed dimensionality reduction results.

The plan is to create a (fixed) reduced dimension plot that will
transmit a multiple selection to our dynamic reduced dimension plot.
This is as easy as:

\begin{Shaded}
\begin{Highlighting}[]
\NormalTok{rdp <-}\StringTok{ }\KeywordTok{ReducedDimensionPlot}\NormalTok{(}\DataTypeTok{PanelId=}\NormalTok{1L)}
\NormalTok{drdp <-}\StringTok{ }\KeywordTok{new}\NormalTok{(}\StringTok{"DynReducedDimensionPlot"}\NormalTok{, }\DataTypeTok{ColumnSelectionSource=}\StringTok{"ReducedDimensionPlot1"}\NormalTok{)}
\NormalTok{app <-}\StringTok{ }\KeywordTok{iSEE}\NormalTok{(sce, }\DataTypeTok{initial=}\KeywordTok{list}\NormalTok{(rdp, drdp))}
\end{Highlighting}
\end{Shaded}

Brushing at any location in \texttt{ReducedDimensionPlot1} will then
trigger dynamically recompution of results in our
\texttt{DynReducedDimensionPlot}.

\chapter{Dynamic differential
expression}\label{dynamic-differential-expression}

\section{Overview}\label{overview-2}

In this case study, we will create a panel class to dynamically compute
differential expression (DE) statistics between the active sample-level
selection and the other saved selections from a transmitting panel. We
will present the results of this computation in a \texttt{DataTable}
widget from the \emph{\href{https://CRAN.R-project.org/package=DT}{DT}}
package, where each row is a gene and each column is a relevant
statistic (\(p\)-value, FDR, log-fold changes, etc.).

\section{Class basics}\label{class-basics-1}

First, we define the basics of our new \texttt{Panel} class. As our new
class will be showing each gene as a row, we inherit from the
\texttt{RowTable} virtual class. This automatically gives us access to
all the functionality promised in the contract, including interface
elements and observers to respond to multiple selections. We also add a
slot specifying the log-fold change threshold to use in the null
hypothesis.

Any new slots should come with validity methods, as shown below.

\begin{Shaded}
\begin{Highlighting}[]
\KeywordTok{library}\NormalTok{(S4Vectors)}
\KeywordTok{setValidity2}\NormalTok{(}\StringTok{"DGETable"}\NormalTok{, }\ControlFlowTok{function}\NormalTok{(object) \{}
\NormalTok{    msg <-}\StringTok{ }\KeywordTok{character}\NormalTok{(}\DecValTok{0}\NormalTok{)}

    \ControlFlowTok{if}\NormalTok{ (}\KeywordTok{length}\NormalTok{(val <-}\StringTok{ }\NormalTok{object[[}\StringTok{"LogFC"}\NormalTok{]])}\OperatorTok{!=}\NormalTok{1L }\OperatorTok{||}\StringTok{ }\NormalTok{val }\OperatorTok{<}\StringTok{ }\DecValTok{0}\NormalTok{) \{}
\NormalTok{        msg <-}\StringTok{ }\KeywordTok{c}\NormalTok{(msg, }\StringTok{"'NGenes' must be a non-negative number"}\NormalTok{)}
\NormalTok{    \}}
    \ControlFlowTok{if}\NormalTok{ (}\KeywordTok{length}\NormalTok{(msg)) \{}
        \KeywordTok{return}\NormalTok{(msg)}
\NormalTok{    \}}
    \OtherTok{TRUE}
\NormalTok{\})}
\end{Highlighting}
\end{Shaded}

It is also worthwhile specializing the \texttt{initialize()} method to
provide a default for new parameters. We hard-code the
\texttt{ColumnSelectionType} setting as we want to obtain all multiple
selections from the transmitting panel, in order to be able to perform
pairwise DE analyses between the various active and saved selections.
(By comparison, the default of \texttt{"Active"} will only transmit the
current active selection.)

\begin{Shaded}
\begin{Highlighting}[]
\KeywordTok{setMethod}\NormalTok{(}\StringTok{"initialize"}\NormalTok{, }\StringTok{"DGETable"}\NormalTok{, }
    \ControlFlowTok{function}\NormalTok{(.Object, }\DataTypeTok{LogFC=}\DecValTok{0}\NormalTok{, ...) }
\NormalTok{\{}
    \KeywordTok{callNextMethod}\NormalTok{(.Object, }\DataTypeTok{LogFC=}\NormalTok{LogFC, }\DataTypeTok{ColumnSelectionType=}\StringTok{"Union"}\NormalTok{, ...)}
\NormalTok{\})}
\end{Highlighting}
\end{Shaded}

\section{Setting up the interface}\label{setting-up-the-interface-1}

The most basic requirement is to define some methods that describe our
new panel in the \texttt{iSEE()} interface. This includes defining the
full name and desired default color for display purposes:

\begin{Shaded}
\begin{Highlighting}[]
\KeywordTok{setMethod}\NormalTok{(}\StringTok{".fullName"}\NormalTok{, }\StringTok{"DGETable"}\NormalTok{, }\ControlFlowTok{function}\NormalTok{(x) }\StringTok{"Differential expression table"}\NormalTok{)}

\KeywordTok{setMethod}\NormalTok{(}\StringTok{".panelColor"}\NormalTok{, }\StringTok{"DGETable"}\NormalTok{, }\ControlFlowTok{function}\NormalTok{(x) }\StringTok{"#55AA00"}\NormalTok{)}
\end{Highlighting}
\end{Shaded}

We also add interface elements to change the result type and the number
of genes. This is most easily done by specializing the
\texttt{.defineDataInterface} method:

\begin{Shaded}
\begin{Highlighting}[]
\KeywordTok{library}\NormalTok{(shiny)}
\KeywordTok{setMethod}\NormalTok{(}\StringTok{".defineDataInterface"}\NormalTok{, }\StringTok{"DGETable"}\NormalTok{, }\ControlFlowTok{function}\NormalTok{(x, se, select_info) \{}
\NormalTok{    plot_name <-}\StringTok{ }\KeywordTok{.getEncodedName}\NormalTok{(x)}
    \KeywordTok{list}\NormalTok{(}
        \KeywordTok{numericInput}\NormalTok{(}\KeywordTok{paste0}\NormalTok{(plot_name, }\StringTok{"_LogFC"}\NormalTok{), }\DataTypeTok{label=}\StringTok{"Log-FC threshold"}\NormalTok{,}
            \DataTypeTok{min=}\DecValTok{0}\NormalTok{, }\DataTypeTok{value=}\NormalTok{x[[}\StringTok{"LogFC"}\NormalTok{]])}
\NormalTok{    )}
\NormalTok{\})}
\end{Highlighting}
\end{Shaded}

As we discussed before, we \texttt{paste0} the name of our panel to the
name of any parameter to ensure that the ID is unique to this instance
of our panel.

\section{Creating the observers}\label{creating-the-observers-1}

We specialize \texttt{.createObservers} to define some observers to
respond to changes in our new interface elements. Note the use of
\texttt{callNextMethod()} to ensure that observers of the parent class
are also created; this automatically ensures that we can respond to
changes in parameters provided by \texttt{RowTable}.

\begin{Shaded}
\begin{Highlighting}[]
\KeywordTok{setMethod}\NormalTok{(}\StringTok{".createObservers"}\NormalTok{, }\StringTok{"DGETable"}\NormalTok{, }
    \ControlFlowTok{function}\NormalTok{(x, se, input, session, pObjects, rObjects) }
\NormalTok{\{}
    \KeywordTok{callNextMethod}\NormalTok{()}

\NormalTok{    plot_name <-}\StringTok{ }\KeywordTok{.getEncodedName}\NormalTok{(x)}

    \KeywordTok{.createUnprotectedParameterObservers}\NormalTok{(plot_name,}
        \DataTypeTok{fields=}\StringTok{"LogFC"}\NormalTok{,}
        \DataTypeTok{input=}\NormalTok{input, }\DataTypeTok{pObjects=}\NormalTok{pObjects, }\DataTypeTok{rObjects=}\NormalTok{rObjects)}
\NormalTok{\})}
\end{Highlighting}
\end{Shaded}

The distinction between protected and unprotected parameters is less
important for \texttt{Table}s; as long as the types of the columns do
not change between renderings, any column or global selections (i.e.,
search terms) are usually still sensible.

\section{Making the table}\label{making-the-table}

When working with a \texttt{RowTable} subclass, the easiest way to
change plotting content to override the \texttt{.generateTable} method.
This is expected to generate a \texttt{data.frame} in the evaluation
environment, returning the commands required to do so. In this case, we
want to perform one-sided \(t\)-tests between the active selection and
any number of saved selections. We will use the \texttt{findMarkers()}
function from
\emph{\href{https://bioconductor.org/packages/3.11/scran}{scran}} to
compute the desired statistics. This performs all pairwise comparisons,
so is not as efficient as could be, but it will suffice for this
demonstration.

\begin{Shaded}
\begin{Highlighting}[]
\KeywordTok{setMethod}\NormalTok{(}\StringTok{".generateTable"}\NormalTok{, }\StringTok{"DGETable"}\NormalTok{, }\ControlFlowTok{function}\NormalTok{(x, envir) \{}
\NormalTok{    empty <-}\StringTok{ "tab <- data.frame(Top=integer(0), p.value=numeric(0), FDR=numeric(0));"}

    \ControlFlowTok{if}\NormalTok{ (}\OperatorTok{!}\KeywordTok{exists}\NormalTok{(}\StringTok{"col_selected"}\NormalTok{, envir, }\DataTypeTok{inherits=}\OtherTok{FALSE}\NormalTok{) }\OperatorTok{||}\StringTok{ }
\StringTok{        }\KeywordTok{length}\NormalTok{(envir}\OperatorTok{$}\NormalTok{col_selected)}\OperatorTok{<}\NormalTok{2L }\OperatorTok{||}\StringTok{ }
\StringTok{        }\OperatorTok{!}\StringTok{"active"} \OperatorTok\StringTok{ }\KeywordTok{names}\NormalTok{(envir}\OperatorTok{$}\NormalTok{col_selected)) }
\NormalTok{    \{}
\NormalTok{        commands <-}\StringTok{ }\NormalTok{empty}
\NormalTok{    \} }\ControlFlowTok{else}\NormalTok{ \{}
\NormalTok{        commands <-}\StringTok{ }\KeywordTok{c}\NormalTok{(}\StringTok{".chosen <- unlist(col_selected);"}\NormalTok{,}
            \StringTok{".grouping <- rep(names(col_selected), lengths(col_selected));"}\NormalTok{,}
            \KeywordTok{sprintf}\NormalTok{(}\StringTok{".de.stats <- scran::findMarkers(logcounts(se)[,.chosen], }
\StringTok{    .grouping, direction='up', lfc=%s)"}\NormalTok{, x[[}\StringTok{"LogFC"}\NormalTok{]]),}
            \StringTok{"tab <- as.data.frame(.de.stats[['active']]);"}
\NormalTok{        )}
\NormalTok{    \}}
       
    \KeywordTok{eval}\NormalTok{(}\KeywordTok{parse}\NormalTok{(}\DataTypeTok{text=}\NormalTok{commands), }\DataTypeTok{envir=}\NormalTok{envir)}

    \KeywordTok{list}\NormalTok{(}\DataTypeTok{commands=}\NormalTok{commands, }\DataTypeTok{contents=}\NormalTok{envir}\OperatorTok{$}\NormalTok{tab)}
\NormalTok{\})}
\end{Highlighting}
\end{Shaded}

Readers may notice that we prefix internal variables with \texttt{.} in
our commands. This ensures that they do not clash with global variables
created by \texttt{iSEE()} itself (which is not an issue when running
the app, but makes things difficult when the code is reported for
tracking purposes).

\section{Finishing touches}\label{finishing-touches-1}

By default, all \texttt{RowTable}s hide their multiple column selection
parameter choices. This considers the typical use case where
\texttt{RowTable}s respond to a selection of rows, rather than a
selection of columns as in our \texttt{DGETable}. Thus, we need to flip
this around so that the unresponsive row selection parameters are hidden
in the interface while the useful column selection parameters are
visible.

We do so by specializing the \texttt{.hideInterface()} method, which
returns \texttt{TRUE} to indicate that a particular interface element
should be hidden. We do not ``un-hide'' \texttt{ColumnSelectionType} and
\texttt{ColumnSelectionSaved} here; our tests are always performed
between the active versus saved selection, so there is no effect from
choosing the selection type.

\begin{Shaded}
\begin{Highlighting}[]
\KeywordTok{setMethod}\NormalTok{(}\StringTok{".hideInterface"}\NormalTok{, }\StringTok{"DGETable"}\NormalTok{, }\ControlFlowTok{function}\NormalTok{(x, field) \{}
    \ControlFlowTok{if}\NormalTok{ (field }\OperatorTok\StringTok{ }\KeywordTok{c}\NormalTok{(}\StringTok{"RowSelectionSource"}\NormalTok{, }\StringTok{"RowSelectionType"}\NormalTok{, }\StringTok{"RowSelectionSaved"}\NormalTok{)) \{}
        \OtherTok{TRUE}
\NormalTok{    \} }\ControlFlowTok{else} \ControlFlowTok{if}\NormalTok{ (field }\OperatorTok\StringTok{ "ColumnSelectionSource"}\NormalTok{) \{}
        \OtherTok{FALSE}
\NormalTok{    \} }\ControlFlowTok{else}\NormalTok{ \{}
        \KeywordTok{callNextMethod}\NormalTok{()}
\NormalTok{    \}}
\NormalTok{\})}
\end{Highlighting}
\end{Shaded}

A more advanced version of this panel class might consider responding to
a row selection by only performing the DE analysis on the selected
features. In such cases, we would not need to hide
\texttt{RowSelectionSource}, though we will leave that as an exercise
for the curious.

\section{In action}\label{in-action-1}

Let's put our new panel to the test. We use the \texttt{sce} object,
preprocessed in a \protect\hyperlink{developing}{previous chapter},
including some precomputed dimensionality reduction results.

The plan is to create a (fixed) reduced dimension plot that will
transmit to our DGE table. Setting up the iSEE instance is as easy as:

\begin{Shaded}
\begin{Highlighting}[]
\NormalTok{rdp <-}\StringTok{ }\KeywordTok{ReducedDimensionPlot}\NormalTok{(}\DataTypeTok{PanelId=}\NormalTok{1L, }\DataTypeTok{SelectionBoxOpen=}\OtherTok{TRUE}\NormalTok{)}
\NormalTok{dget <-}\StringTok{ }\KeywordTok{new}\NormalTok{(}\StringTok{"DGETable"}\NormalTok{, }\DataTypeTok{ColumnSelectionSource=}\StringTok{"ReducedDimensionPlot1"}\NormalTok{, }\DataTypeTok{PanelWidth=}\NormalTok{8L)}
\NormalTok{app <-}\StringTok{ }\KeywordTok{iSEE}\NormalTok{(sce, }\DataTypeTok{initial=}\KeywordTok{list}\NormalTok{(rdp, dget))}
\end{Highlighting}
\end{Shaded}

Brushing (or lassoing) at any location and saving the selection will
trigger dynamic recompution of results in our \texttt{DGETable}. We can
repeat this with any number of saved selections.

\chapter{Annotated gene list}\label{annotated-gene-list}

\section{Overview}\label{overview-3}

When given a gene list, we often need to look up the function of the top
genes in a search engine. This typically involves copy-pasting the gene
name or ID into the search box and pressing Enter, which is a pain.
Instead, we can automate this process in
\emph{\href{https://bioconductor.org/packages/3.11/iSEE}{iSEE}} by
creating an \textbf{annotated gene table}. We demonstrate by showing how
we can dynamically look up annotation for each gene in the
\texttt{rowData} of a \texttt{SummarizedExperiment}.

\section{Class basics}\label{class-basics-2}

First, we define the basics of our new \texttt{Panel} class. Our new
class will be showing the gene-level metadata, so we inherit from the
\texttt{RowDataTable} class that does exactly this. We add some slots
specifying which column of the table contains our gene IDs, the type of
ID and the organism database to use.

We specialize the validity method to check that the \texttt{IDColumn} is
either a string or \texttt{NULL}; if the latter, we assume that the ID
is stored in the row name. We also add some cursory checks for the other
parameters.

\begin{Shaded}
\begin{Highlighting}[]
\NormalTok{allowable <-}\StringTok{ }\KeywordTok{c}\NormalTok{(}\StringTok{"ENSEMBL"}\NormalTok{, }\StringTok{"SYMBOL"}\NormalTok{, }\StringTok{"ENTREZID"}\NormalTok{)}
\KeywordTok{setValidity2}\NormalTok{(}\StringTok{"GeneAnnoTable"}\NormalTok{, }\ControlFlowTok{function}\NormalTok{(object) \{}
\NormalTok{    msg <-}\StringTok{ }\KeywordTok{character}\NormalTok{(}\DecValTok{0}\NormalTok{)}

    \ControlFlowTok{if}\NormalTok{ (}\OperatorTok{!}\KeywordTok{is.null}\NormalTok{(val <-}\StringTok{ }\NormalTok{object[[}\StringTok{"IDColumn"}\NormalTok{]]) }\OperatorTok{&&}\StringTok{ }\NormalTok{(}\KeywordTok{length}\NormalTok{(val)}\OperatorTok{!=}\NormalTok{1L }\OperatorTok{||}\StringTok{ }\KeywordTok{is.na}\NormalTok{(val))) \{}
\NormalTok{        msg <-}\StringTok{ }\KeywordTok{c}\NormalTok{(msg, }\StringTok{"'IDColumn must be NULL or a string"}\NormalTok{)}
\NormalTok{    \}}

    \ControlFlowTok{if}\NormalTok{ (}\OperatorTok{!}\KeywordTok{isSingleString}\NormalTok{(orgdb <-}\StringTok{ }\NormalTok{object[[}\StringTok{"Organism"}\NormalTok{]])) \{}
\NormalTok{        msg <-}\StringTok{ }\KeywordTok{c}\NormalTok{(msg, }\KeywordTok{sprintf}\NormalTok{(}\StringTok{"'Organism' should be a single string"}\NormalTok{, orgdb))}
\NormalTok{    \}}

    \ControlFlowTok{if}\NormalTok{ (}\OperatorTok{!}\KeywordTok{isSingleString}\NormalTok{(type <-}\StringTok{ }\NormalTok{object[[}\StringTok{"IDType"}\NormalTok{]]) }\OperatorTok{||}\StringTok{ }\OperatorTok{!}\NormalTok{type }\OperatorTok\StringTok{ }\NormalTok{allowable) \{}
\NormalTok{        msg <-}\StringTok{ }\KeywordTok{c}\NormalTok{(msg, }\StringTok{"'IDType' should be 'ENSEMBL', 'SYMBOL' or 'ENTREZID'"}\NormalTok{)}
\NormalTok{    \}}

    \ControlFlowTok{if}\NormalTok{ (}\KeywordTok{length}\NormalTok{(open <-}\StringTok{ }\NormalTok{object[[}\StringTok{"AnnoBoxOpen"}\NormalTok{]])}\OperatorTok{!=}\NormalTok{1L }\OperatorTok{||}\StringTok{ }\KeywordTok{is.na}\NormalTok{(open)) \{}
\NormalTok{        msg <-}\StringTok{ }\KeywordTok{c}\NormalTok{(msg, }\StringTok{"'AnnoBoxOpen' should be a non-missing logical scalar"}\NormalTok{)}
\NormalTok{    \}}

    \ControlFlowTok{if}\NormalTok{ (}\KeywordTok{length}\NormalTok{(msg)) \{}
        \KeywordTok{return}\NormalTok{(msg)}
\NormalTok{    \}}
    \OtherTok{TRUE}
\NormalTok{\})}
\end{Highlighting}
\end{Shaded}

We then specialize the initialize method to set reasonable defaults for
these parameters.

\begin{Shaded}
\begin{Highlighting}[]
\KeywordTok{setMethod}\NormalTok{(}\StringTok{"initialize"}\NormalTok{, }\StringTok{"GeneAnnoTable"}\NormalTok{, }\ControlFlowTok{function}\NormalTok{(.Object, }\DataTypeTok{IDColumn=}\OtherTok{NULL}\NormalTok{, }
    \DataTypeTok{Organism=}\StringTok{"org.Mm.eg.db"}\NormalTok{, }\DataTypeTok{IDType=}\StringTok{"SYMBOL"}\NormalTok{, }\DataTypeTok{AnnoBoxOpen=}\OtherTok{FALSE}\NormalTok{, ...)}
\NormalTok{\{}
    \KeywordTok{callNextMethod}\NormalTok{(.Object, }\DataTypeTok{IDColumn=}\NormalTok{IDColumn, }\DataTypeTok{IDType=}\NormalTok{IDType,}
        \DataTypeTok{Organism=}\NormalTok{Organism, }\DataTypeTok{AnnoBoxOpen=}\NormalTok{AnnoBoxOpen, ...)}
\NormalTok{\})}
\end{Highlighting}
\end{Shaded}

\section{Setting up the interface}\label{setting-up-the-interface-2}

We define the full name and desired default color for display purposes:

\begin{Shaded}
\begin{Highlighting}[]
\KeywordTok{setMethod}\NormalTok{(}\StringTok{".fullName"}\NormalTok{, }\StringTok{"GeneAnnoTable"}\NormalTok{, }\ControlFlowTok{function}\NormalTok{(x) }\StringTok{"Annotated gene table"}\NormalTok{)}

\KeywordTok{setMethod}\NormalTok{(}\StringTok{".panelColor"}\NormalTok{, }\StringTok{"GeneAnnoTable"}\NormalTok{, }\ControlFlowTok{function}\NormalTok{(x) }\StringTok{"#AA1122"}\NormalTok{)}
\end{Highlighting}
\end{Shaded}

We want to add another UI element for showing the gene-level annotation.
This is achieved by specializing the \texttt{.defineOutput()} method as
shown below; note the prefixing by the panel name to ensure that output
element IDs from different panels are unique.

\begin{Shaded}
\begin{Highlighting}[]
\KeywordTok{setMethod}\NormalTok{(}\StringTok{".defineOutput"}\NormalTok{, }\StringTok{"GeneAnnoTable"}\NormalTok{, }\ControlFlowTok{function}\NormalTok{(x, ...) \{}
\NormalTok{    panel_name <-}\StringTok{ }\KeywordTok{.getEncodedName}\NormalTok{(x)}
    \KeywordTok{tagList}\NormalTok{(}
        \KeywordTok{callNextMethod}\NormalTok{(), }\CommentTok{# Re-using RowDataTable's definition.}
        \KeywordTok{uiOutput}\NormalTok{(}\KeywordTok{paste0}\NormalTok{(panel_name, }\StringTok{"_annotation"}\NormalTok{)),}
        \KeywordTok{hr}\NormalTok{()}
\NormalTok{    )}
\NormalTok{\})}
\end{Highlighting}
\end{Shaded}

We also set up interface elements for changing the annotation
parameters. We will put these elements in a separate ``Annotation
parameters'' collapsible box, which is initialized in an opened or
closed state depending on the \texttt{AnnoBoxOpen} slot.

\begin{Shaded}
\begin{Highlighting}[]
\KeywordTok{setMethod}\NormalTok{(}\StringTok{".defineInterface"}\NormalTok{, }\StringTok{"GeneAnnoTable"}\NormalTok{, }\ControlFlowTok{function}\NormalTok{(x, se, select_info) \{}
\NormalTok{    panel_name <-}\StringTok{ }\KeywordTok{.getEncodedName}\NormalTok{(x)}
    \KeywordTok{c}\NormalTok{(}
        \KeywordTok{list}\NormalTok{(}
            \KeywordTok{collapseBox}\NormalTok{(}
                \KeywordTok{paste0}\NormalTok{(panel_name, }\StringTok{"_AnnoBoxOpen"}\NormalTok{),}
                \DataTypeTok{title=}\StringTok{"Annotation parameters"}\NormalTok{,}
                \DataTypeTok{open=}\NormalTok{x[[}\StringTok{"AnnoBoxOpen"}\NormalTok{]],}
                \KeywordTok{selectInput}\NormalTok{(}\KeywordTok{paste0}\NormalTok{(panel_name, }\StringTok{"_IDColumn"}\NormalTok{),}
                    \DataTypeTok{label=}\StringTok{"ID-containing column:"}\NormalTok{,}
                    \DataTypeTok{choices=}\KeywordTok{colnames}\NormalTok{(}\KeywordTok{rowData}\NormalTok{(se)), }
                    \DataTypeTok{selected=}\NormalTok{x[[}\StringTok{"IDColumn"}\NormalTok{]]}
\NormalTok{                ),}
                \KeywordTok{selectInput}\NormalTok{(}\KeywordTok{paste0}\NormalTok{(panel_name, }\StringTok{"_IDType"}\NormalTok{),}
                    \DataTypeTok{label=}\StringTok{"ID type:"}\NormalTok{,}
                    \DataTypeTok{choices=}\NormalTok{allowable,}
                    \DataTypeTok{selected=}\NormalTok{x[[}\StringTok{"IDType"}\NormalTok{]]}
\NormalTok{                ),}
                \KeywordTok{selectInput}\NormalTok{(}\KeywordTok{paste0}\NormalTok{(panel_name, }\StringTok{"_Organism"}\NormalTok{),}
                    \DataTypeTok{label=}\StringTok{"Organism"}\NormalTok{,}
                    \DataTypeTok{choices=}\KeywordTok{c}\NormalTok{(}\StringTok{"org.Hs.eg.db"}\NormalTok{, }\StringTok{"org.Mm.eg.db"}\NormalTok{),}
                    \DataTypeTok{selected=}\NormalTok{x[[}\StringTok{"Organism"}\NormalTok{]]}
\NormalTok{                )}
\NormalTok{            )}
\NormalTok{        ),}
        \KeywordTok{callNextMethod}\NormalTok{()}
\NormalTok{    ) }
\NormalTok{\})}
\end{Highlighting}
\end{Shaded}

\section{Creating the observers}\label{creating-the-observers-2}

We specialize \texttt{.createObservers} to define some observers to
respond to changes in our new interface elements. Note the use of
\texttt{callNextMethod()} to ensure that observers of the parent class
are also created.

\begin{Shaded}
\begin{Highlighting}[]
\KeywordTok{setMethod}\NormalTok{(}\StringTok{".createObservers"}\NormalTok{, }\StringTok{"GeneAnnoTable"}\NormalTok{,}
    \ControlFlowTok{function}\NormalTok{(x, se, input, session, pObjects, rObjects)}
\NormalTok{\{}
    \KeywordTok{callNextMethod}\NormalTok{()}

\NormalTok{    plot_name <-}\StringTok{ }\KeywordTok{.getEncodedName}\NormalTok{(x)}

    \KeywordTok{.createUnprotectedParameterObservers}\NormalTok{(plot_name,}
        \DataTypeTok{fields=}\KeywordTok{c}\NormalTok{(}\StringTok{"IDColumn"}\NormalTok{, }\StringTok{"Organism"}\NormalTok{, }\StringTok{"IDType"}\NormalTok{), }
        \DataTypeTok{input=}\NormalTok{input, }\DataTypeTok{pObjects=}\NormalTok{pObjects, }\DataTypeTok{rObjects=}\NormalTok{rObjects)}
\NormalTok{\})}
\end{Highlighting}
\end{Shaded}

We need to set up a rendering expression for the annotation element that
responds to the selected gene. By using
\texttt{.trackSingleSelection()}, we ensure that this UI element updates
in response to changes in the table selection. We add a series of
protective measures to avoid the application crashing due to missing
organism packages or unmatched IDs.

\begin{Shaded}
\begin{Highlighting}[]
\KeywordTok{setMethod}\NormalTok{(}\StringTok{".renderOutput"}\NormalTok{, }\StringTok{"GeneAnnoTable"}\NormalTok{, }\ControlFlowTok{function}\NormalTok{(x, se, ..., output, pObjects, rObjects) \{}
    \KeywordTok{callNextMethod}\NormalTok{() }\CommentTok{# Re-using RowDataTable's output rendering.}

\NormalTok{    panel_name <-}\StringTok{ }\KeywordTok{.getEncodedName}\NormalTok{(x)}
\NormalTok{    output[[}\KeywordTok{paste0}\NormalTok{(panel_name, }\StringTok{"_annotation"}\NormalTok{)]] <-}\StringTok{ }\KeywordTok{renderUI}\NormalTok{(\{}
        \KeywordTok{.trackSingleSelection}\NormalTok{(panel_name, rObjects)}
\NormalTok{        instance <-}\StringTok{ }\NormalTok{pObjects}\OperatorTok{$}\NormalTok{memory[[panel_name]]}

\NormalTok{        rowdata_col <-}\StringTok{ }\NormalTok{instance[[}\StringTok{"IDColumn"}\NormalTok{]]}
\NormalTok{        selectedGene <-}\StringTok{ }\NormalTok{instance[[}\StringTok{"Selected"}\NormalTok{]]}
        \ControlFlowTok{if}\NormalTok{ (}\OperatorTok{!}\KeywordTok{is.null}\NormalTok{(rowdata_col)) \{}
\NormalTok{            selectedGene <-}\StringTok{ }\KeywordTok{rowData}\NormalTok{(se)[selectedGene,rowdata_col]}
\NormalTok{        \}}

\NormalTok{        keytype <-}\StringTok{ }\NormalTok{instance[[}\StringTok{"IDType"}\NormalTok{]]}
\NormalTok{        selgene_entrez <-}\StringTok{ }\OtherTok{NA}
        \ControlFlowTok{if}\NormalTok{ (keytype}\OperatorTok{!=}\StringTok{"ENTREZID"}\NormalTok{) \{}
\NormalTok{            ORG <-}\StringTok{ }\NormalTok{instance[[}\StringTok{"Organism"}\NormalTok{]]}
            \ControlFlowTok{if}\NormalTok{ (}\KeywordTok{require}\NormalTok{(ORG, }\DataTypeTok{character.only=}\OtherTok{TRUE}\NormalTok{, }\DataTypeTok{quietly=}\OtherTok{TRUE}\NormalTok{)) \{}
\NormalTok{                orgdb <-}\StringTok{ }\KeywordTok{get}\NormalTok{(ORG)}
\NormalTok{                selgene_entrez <-}\StringTok{ }\KeywordTok{try}\NormalTok{(}\KeywordTok{mapIds}\NormalTok{(orgdb, selectedGene, }\StringTok{"ENTREZID"}\NormalTok{, keytype), }
                    \DataTypeTok{silent=}\OtherTok{TRUE}\NormalTok{)}
\NormalTok{            \}}
\NormalTok{        \} }\ControlFlowTok{else}\NormalTok{ \{}
\NormalTok{            selgene_entrez <-}\StringTok{ }\NormalTok{selectedGene}
\NormalTok{        \}}

        \ControlFlowTok{if}\NormalTok{ (}\KeywordTok{is.na}\NormalTok{(selgene_entrez) }\OperatorTok{||}\StringTok{ }\KeywordTok{is}\NormalTok{(selgene_entrez, }\StringTok{"try-error"}\NormalTok{)) \{}
            \KeywordTok{return}\NormalTok{(}\OtherTok{NULL}\NormalTok{)}
\NormalTok{        \}}

\NormalTok{        fullinfo <-}\StringTok{ }\NormalTok{rentrez}\OperatorTok{::}\KeywordTok{entrez_summary}\NormalTok{(}\StringTok{"gene"}\NormalTok{, selgene_entrez)}
\NormalTok{        link_pubmed <-}\StringTok{ }\KeywordTok{paste0}\NormalTok{(}\StringTok{'<a href="http://www.ncbi.nlm.nih.gov/gene/?term='}\NormalTok{,}
\NormalTok{            selgene_entrez,}
            \StringTok{'" target="_blank">Click here to see more at the NCBI database</a>'}\NormalTok{)}

\NormalTok{        mycontent <-}\StringTok{ }\KeywordTok{paste0}\NormalTok{(}\StringTok{"<b>"}\NormalTok{,fullinfo}\OperatorTok{$}\NormalTok{name, }\StringTok{"</b><br/><br/>"}\NormalTok{,}
\NormalTok{            fullinfo}\OperatorTok{$}\NormalTok{description,}\StringTok{"<br/><br/>"}\NormalTok{,}
            \KeywordTok{ifelse}\NormalTok{(fullinfo}\OperatorTok{$}\NormalTok{summary }\OperatorTok{==}\StringTok{ ""}\NormalTok{,}\StringTok{""}\NormalTok{,}\KeywordTok{paste0}\NormalTok{(fullinfo}\OperatorTok{$}\NormalTok{summary, }\StringTok{"<br/><br/>"}\NormalTok{)),}
\NormalTok{            link_pubmed)}

        \KeywordTok{HTML}\NormalTok{(mycontent)}
\NormalTok{    \})}
\NormalTok{\})}
\end{Highlighting}
\end{Shaded}

Observant readers will note that the body of the rendering expression
uses \texttt{instance} rather than \texttt{x}. This is intentional as it
ensures that we are using the parameter settings from the current state
of the app. If we used \texttt{x}, we would always be using the
parameters from the initial state of the app, which is not what we want.

\section{In action}\label{in-action-2}

Let's put our new panel to the test using the \texttt{sce} object,
preprocessed in a \protect\hyperlink{developing}{previous chapter}.

Setting up the iSEE instance is as easy as:

\begin{Shaded}
\begin{Highlighting}[]
\NormalTok{gat <-}\StringTok{ }\KeywordTok{new}\NormalTok{(}\StringTok{"GeneAnnoTable"}\NormalTok{, }\DataTypeTok{PanelWidth=}\NormalTok{8L)}
\NormalTok{app <-}\StringTok{ }\KeywordTok{iSEE}\NormalTok{(sce, }\DataTypeTok{initial=}\KeywordTok{list}\NormalTok{(gat))}
\end{Highlighting}
\end{Shaded}

Clicking on any row will bring up the Entrez annotation (if available)
for that feature. It is probably best to click on some well-annotated
genes as the set of RIKEN transcripts at the front don't have much
annotation.

\chapter{Gene ontology table}\label{gene-ontology-table}

\section{Overview}\label{overview-4}

Here, we will construct a table of GO terms where selection of a row in
the table causes transmission of a multiple selection of gene names. The
aim is to enable us to transmit multiple row selections to other panels
based on their membership of a gene set. This is a fairly involved
example of creating a \texttt{Panel} subclass as we cannot easily
inherit from an existing subclass; rather, we need to provide all the
methods ourselves.

\section{Class basics}\label{class-basics-3}

First, we define the basics of our new \texttt{GOTable} class. This
inherits from the virtual base \texttt{Panel} class as it cannot meet
any of the contractual requirements of the subclasses, what with the
\texttt{DataTable} selection event triggering a multiple selection
rather than a single selection. We add some slots to specify the feature
ID type and the organism of interest as well as for \texttt{DataTable}
parameters.

We also add some checks for these parameters.

\begin{Shaded}
\begin{Highlighting}[]
\NormalTok{allowable <-}\StringTok{ }\KeywordTok{c}\NormalTok{(}\StringTok{"ENSEMBL"}\NormalTok{, }\StringTok{"SYMBOL"}\NormalTok{, }\StringTok{"ENTREZID"}\NormalTok{)}
\KeywordTok{setValidity2}\NormalTok{(}\StringTok{"GOTable"}\NormalTok{, }\ControlFlowTok{function}\NormalTok{(object) \{}
\NormalTok{    msg <-}\StringTok{ }\KeywordTok{character}\NormalTok{(}\DecValTok{0}\NormalTok{)}

    \ControlFlowTok{if}\NormalTok{ (}\OperatorTok{!}\KeywordTok{isSingleString}\NormalTok{(orgdb <-}\StringTok{ }\NormalTok{object[[}\StringTok{"Organism"}\NormalTok{]])) \{}
\NormalTok{        msg <-}\StringTok{ }\KeywordTok{c}\NormalTok{(msg, }\KeywordTok{sprintf}\NormalTok{(}\StringTok{"'Organism' should be a single string"}\NormalTok{, orgdb))}
\NormalTok{    \}}

    \ControlFlowTok{if}\NormalTok{ (}\OperatorTok{!}\KeywordTok{isSingleString}\NormalTok{(type <-}\StringTok{ }\NormalTok{object[[}\StringTok{"IDType"}\NormalTok{]]) }\OperatorTok{||}\StringTok{ }\OperatorTok{!}\NormalTok{type }\OperatorTok\StringTok{ }\NormalTok{allowable) \{}
\NormalTok{        msg <-}\StringTok{ }\KeywordTok{c}\NormalTok{(msg, }\StringTok{"'IDType' should be 'ENSEMBL', 'SYMBOL' or 'ENTREZID'"}\NormalTok{)}
\NormalTok{    \}}

    \ControlFlowTok{if}\NormalTok{ (}\OperatorTok{!}\KeywordTok{isSingleString}\NormalTok{(object[[}\StringTok{"Selected"}\NormalTok{]])) \{ }
\NormalTok{        msg <-}\StringTok{ }\KeywordTok{c}\NormalTok{(msg, }\StringTok{"'Selected' should be a single string"}\NormalTok{)}
\NormalTok{    \}}

    \ControlFlowTok{if}\NormalTok{ (}\OperatorTok{!}\KeywordTok{isSingleString}\NormalTok{(object[[}\StringTok{"Search"}\NormalTok{]])) \{}
\NormalTok{        msg <-}\StringTok{ }\KeywordTok{c}\NormalTok{(msg, }\StringTok{"'Search' should be a single string"}\NormalTok{)}
\NormalTok{    \}}

    \ControlFlowTok{if}\NormalTok{ (}\KeywordTok{length}\NormalTok{(msg)) \{}
        \KeywordTok{return}\NormalTok{(msg)}
\NormalTok{    \}}
    \OtherTok{TRUE}
\NormalTok{\})}
\end{Highlighting}
\end{Shaded}

We then specialize the initialize method to set reasonable defaults.

\begin{Shaded}
\begin{Highlighting}[]
\KeywordTok{setMethod}\NormalTok{(}\StringTok{"initialize"}\NormalTok{, }\StringTok{"GOTable"}\NormalTok{, }\ControlFlowTok{function}\NormalTok{(.Object, }
    \DataTypeTok{Organism=}\StringTok{"org.Mm.eg.db"}\NormalTok{, }\DataTypeTok{IDType=}\StringTok{"SYMBOL"}\NormalTok{, }
    \DataTypeTok{Selected=}\StringTok{""}\NormalTok{, }\DataTypeTok{Search=}\StringTok{""}\NormalTok{, }\DataTypeTok{SearchColumns=}\KeywordTok{character}\NormalTok{(}\DecValTok{0}\NormalTok{), ...)}
\NormalTok{\{}
    \KeywordTok{callNextMethod}\NormalTok{(.Object, }\DataTypeTok{IDType=}\NormalTok{IDType, }\DataTypeTok{Organism=}\NormalTok{Organism, }
        \DataTypeTok{Selected=}\NormalTok{Selected, }\DataTypeTok{Search=}\NormalTok{Search, }
        \DataTypeTok{SearchColumns=}\NormalTok{SearchColumns, ...)}
\NormalTok{\})}
\end{Highlighting}
\end{Shaded}

\section{Setting up the interface}\label{setting-up-the-interface-3}

We define the full name and desired default color for display purposes:

\begin{Shaded}
\begin{Highlighting}[]
\KeywordTok{setMethod}\NormalTok{(}\StringTok{".fullName"}\NormalTok{, }\StringTok{"GOTable"}\NormalTok{, }\ControlFlowTok{function}\NormalTok{(x) }\StringTok{"Gene ontology table"}\NormalTok{)}

\KeywordTok{setMethod}\NormalTok{(}\StringTok{".panelColor"}\NormalTok{, }\StringTok{"GOTable"}\NormalTok{, }\ControlFlowTok{function}\NormalTok{(x) }\StringTok{"#BB00FF"}\NormalTok{)}
\end{Highlighting}
\end{Shaded}

We add our UI element for showing the gene set table, which is simply a
\texttt{DataTable} object from the
\emph{\href{https://CRAN.R-project.org/package=DT}{DT}} package. Note
that \emph{\href{https://CRAN.R-project.org/package=shiny}{shiny}} also
has a \texttt{dataTableOutput} function so care must be taken to
disambiguate them.

\begin{Shaded}
\begin{Highlighting}[]
\KeywordTok{setMethod}\NormalTok{(}\StringTok{".defineOutput"}\NormalTok{, }\StringTok{"GOTable"}\NormalTok{, }\ControlFlowTok{function}\NormalTok{(x, ...) \{}
\NormalTok{    panel_name <-}\StringTok{ }\KeywordTok{.getEncodedName}\NormalTok{(x)}
    \KeywordTok{tagList}\NormalTok{(DT}\OperatorTok{::}\KeywordTok{dataTableOutput}\NormalTok{(panel_name))}
\NormalTok{\})}
\end{Highlighting}
\end{Shaded}

We set up interface elements for changing the annotation parameters.

\begin{Shaded}
\begin{Highlighting}[]
\KeywordTok{setMethod}\NormalTok{(}\StringTok{".defineDataInterface"}\NormalTok{, }\StringTok{"GOTable"}\NormalTok{, }\ControlFlowTok{function}\NormalTok{(x, se, select_info) \{}
\NormalTok{    panel_name <-}\StringTok{ }\KeywordTok{.getEncodedName}\NormalTok{(x)}
    \KeywordTok{list}\NormalTok{(}
        \KeywordTok{selectInput}\NormalTok{(}\KeywordTok{paste0}\NormalTok{(panel_name, }\StringTok{"_IDType"}\NormalTok{),}
            \DataTypeTok{label=}\StringTok{"ID type:"}\NormalTok{,}
            \DataTypeTok{choices=}\NormalTok{allowable,}
            \DataTypeTok{selected=}\NormalTok{x[[}\StringTok{"IDType"}\NormalTok{]]}
\NormalTok{        ),}
        \KeywordTok{selectInput}\NormalTok{(}\KeywordTok{paste0}\NormalTok{(panel_name, }\StringTok{"_Organism"}\NormalTok{),}
            \DataTypeTok{label=}\StringTok{"Organism"}\NormalTok{,}
            \DataTypeTok{choices=}\KeywordTok{c}\NormalTok{(}\StringTok{"org.Hs.eg.db"}\NormalTok{, }\StringTok{"org.Mm.eg.db"}\NormalTok{),}
            \DataTypeTok{selected=}\NormalTok{x[[}\StringTok{"Organism"}\NormalTok{]]}
\NormalTok{        )}
\NormalTok{    )}
\NormalTok{\})}
\end{Highlighting}
\end{Shaded}

Our implementation will be a pure transmitter, i.e., it will not respond
to row or column selections from other panels. To avoid confusion, we
can hide all selection parameter UI elements by specializing the
\texttt{.hideInterface()} method:

\begin{Shaded}
\begin{Highlighting}[]
\KeywordTok{setMethod}\NormalTok{(}\StringTok{".hideInterface"}\NormalTok{, }\StringTok{"GOTable"}\NormalTok{, }\ControlFlowTok{function}\NormalTok{(x, field) \{}
    \ControlFlowTok{if}\NormalTok{ (field }\OperatorTok\StringTok{ "SelectionBoxOpen"}\NormalTok{) \{}
        \OtherTok{TRUE}
\NormalTok{    \} }\ControlFlowTok{else}\NormalTok{ \{}
        \KeywordTok{callNextMethod}\NormalTok{()}
\NormalTok{    \}}
\NormalTok{\})}
\end{Highlighting}
\end{Shaded}

\section{Generating the output}\label{generating-the-output}

We actually generate the output by specializing the
\texttt{.generateOutput()} function, using the
\emph{\href{https://bioconductor.org/packages/3.11/GO.db}{GO.db}}
package to create a table of GO terms and their definitions. We also
store the number of available genes in the \texttt{contents} - this will
be used later to compute the percentage of all genes in a given gene
set.

\begin{Shaded}
\begin{Highlighting}[]
\KeywordTok{setMethod}\NormalTok{(}\StringTok{".generateOutput"}\NormalTok{, }\StringTok{"GOTable"}\NormalTok{, }\ControlFlowTok{function}\NormalTok{(x, se, ..., all_memory, all_contents) \{}
\NormalTok{    envir <-}\StringTok{ }\KeywordTok{new.env}\NormalTok{()}
\NormalTok{    commands <-}\StringTok{ }\KeywordTok{c}\NormalTok{(}\StringTok{"require(GO.db);"}\NormalTok{,}
        \StringTok{"tab <- select(GO.db, keys=keys(GO.db), columns='TERM');"}\NormalTok{,}
        \StringTok{"rownames(tab) <- tab$GOID;"}\NormalTok{,}
        \StringTok{"tab$GOID <- NULL;"}\NormalTok{)}
    \KeywordTok{eval}\NormalTok{(}\KeywordTok{parse}\NormalTok{(}\DataTypeTok{text=}\NormalTok{commands), }\DataTypeTok{envir=}\NormalTok{envir)}
    \KeywordTok{list}\NormalTok{(}
        \DataTypeTok{commands=}\KeywordTok{list}\NormalTok{(commands), }
        \DataTypeTok{contents=}\KeywordTok{list}\NormalTok{(}\DataTypeTok{table=}\NormalTok{envir}\OperatorTok{$}\NormalTok{tab, }\DataTypeTok{available=}\KeywordTok{nrow}\NormalTok{(se))}
\NormalTok{    )}
\NormalTok{\})}
\end{Highlighting}
\end{Shaded}

We don't actually depend on any parameters of \texttt{x} itself to
generate this table. However, one could imagine a more complex case
where the \texttt{GOTable} itself responds to a multiple row selection,
e.g., by subsetting to the gene sets that contain genes in the selected
row.

\section{Creating the observers}\label{creating-the-observers-3}

We specialize \texttt{.createObservers} to define some observers to
respond to changes in our new interface elements. This also involves
creating an observer to respond to a change in the selection of a
\texttt{DataTable} row, calling \texttt{.requestActiveSelectionUpdate()}
to trigger changes in panels that are receiving the multiple row
selection. (We set up observers for the search fields as well, as a
courtesy to restore them properly upon re-rendering.) Note the use of
\texttt{callNextMethod()} to ensure that observers of the parent class
are also created.

\begin{Shaded}
\begin{Highlighting}[]
\KeywordTok{setMethod}\NormalTok{(}\StringTok{".createObservers"}\NormalTok{, }\StringTok{"GOTable"}\NormalTok{,}
    \ControlFlowTok{function}\NormalTok{(x, se, input, session, pObjects, rObjects)}
\NormalTok{\{}
    \KeywordTok{callNextMethod}\NormalTok{()}

\NormalTok{    panel_name <-}\StringTok{ }\KeywordTok{.getEncodedName}\NormalTok{(x)}

    \KeywordTok{.createUnprotectedParameterObservers}\NormalTok{(panel_name,}
        \DataTypeTok{fields=}\KeywordTok{c}\NormalTok{(}\StringTok{"Organism"}\NormalTok{, }\StringTok{"IDType"}\NormalTok{), }
        \DataTypeTok{input=}\NormalTok{input, }\DataTypeTok{pObjects=}\NormalTok{pObjects, }\DataTypeTok{rObjects=}\NormalTok{rObjects)}

    \CommentTok{# Observer for the DataTable row selection:}
\NormalTok{    select_field <-}\StringTok{ }\KeywordTok{paste0}\NormalTok{(panel_name, }\StringTok{"_rows_selected"}\NormalTok{)}
\NormalTok{    multi_name <-}\StringTok{ }\KeywordTok{paste0}\NormalTok{(panel_name, }\StringTok{"_"}\NormalTok{, iSEE}\OperatorTok{:::}\NormalTok{.flagMultiSelect)}
    \KeywordTok{observeEvent}\NormalTok{(input[[select_field]], \{}
\NormalTok{        chosen <-}\StringTok{ }\NormalTok{input[[select_field]]}
        \ControlFlowTok{if}\NormalTok{ (}\KeywordTok{length}\NormalTok{(chosen)}\OperatorTok{==}\NormalTok{0L) \{}
\NormalTok{            chosen <-}\StringTok{ ""} 
\NormalTok{        \} }\ControlFlowTok{else}\NormalTok{ \{}
\NormalTok{            chosen <-}\StringTok{ }\KeywordTok{rownames}\NormalTok{(pObjects}\OperatorTok{$}\NormalTok{contents[[panel_name]]}\OperatorTok{$}\NormalTok{table)[chosen]}
\NormalTok{        \}}

\NormalTok{        previous <-}\StringTok{ }\NormalTok{pObjects}\OperatorTok{$}\NormalTok{memory[[panel_name]][[}\StringTok{"Selected"}\NormalTok{]]}
        \ControlFlowTok{if}\NormalTok{ (chosen}\OperatorTok{==}\NormalTok{previous) \{}
            \KeywordTok{return}\NormalTok{(}\OtherTok{NULL}\NormalTok{)}
\NormalTok{        \}}
\NormalTok{        pObjects}\OperatorTok{$}\NormalTok{memory[[panel_name]][[}\StringTok{"Selected"}\NormalTok{]] <-}\StringTok{ }\NormalTok{chosen}
        \KeywordTok{.requestActiveSelectionUpdate}\NormalTok{(panel_name, rObjects, }\DataTypeTok{update_output=}\OtherTok{FALSE}\NormalTok{)}
\NormalTok{    \}, }\DataTypeTok{ignoreNULL=}\OtherTok{FALSE}\NormalTok{)}

    \CommentTok{# Observer for the search field:}
\NormalTok{    search_field <-}\StringTok{ }\KeywordTok{paste0}\NormalTok{(panel_name, }\StringTok{"_search"}\NormalTok{)}
    \KeywordTok{observeEvent}\NormalTok{(input[[search_field]], \{}
\NormalTok{        search <-}\StringTok{ }\NormalTok{input[[search_field]]}
        \ControlFlowTok{if}\NormalTok{ (}\KeywordTok{identical}\NormalTok{(search, pObjects}\OperatorTok{$}\NormalTok{memory[[panel_name]][[}\StringTok{"Search"}\NormalTok{]])) \{}
            \KeywordTok{return}\NormalTok{(}\OtherTok{NULL}\NormalTok{)}
\NormalTok{        \}}
\NormalTok{        pObjects}\OperatorTok{$}\NormalTok{memory[[panel_name]][[}\StringTok{"Search"}\NormalTok{]] <-}\StringTok{ }\NormalTok{search}
\NormalTok{    \})}

    \CommentTok{# Observer for the column search fields:}
\NormalTok{    colsearch_field <-}\StringTok{ }\KeywordTok{paste0}\NormalTok{(panel_name, }\StringTok{"_search_columns"}\NormalTok{)}
    \KeywordTok{observeEvent}\NormalTok{(input[[colsearch_field]], \{}
\NormalTok{        search <-}\StringTok{ }\NormalTok{input[[colsearch_field]]}
        \ControlFlowTok{if}\NormalTok{ (}\KeywordTok{identical}\NormalTok{(search, pObjects}\OperatorTok{$}\NormalTok{memory[[panel_name]][[}\StringTok{"SearchColumns"}\NormalTok{]])) \{}
            \KeywordTok{return}\NormalTok{(}\OtherTok{NULL}\NormalTok{)}
\NormalTok{        \}}
\NormalTok{        pObjects}\OperatorTok{$}\NormalTok{memory[[panel_name]][[}\StringTok{"SearchColumns"}\NormalTok{]] <-}\StringTok{ }\NormalTok{search}
\NormalTok{    \})}
\NormalTok{\})}
\end{Highlighting}
\end{Shaded}

We set up a rendering expression for the output table by specializing
\texttt{.renderOutput()}. This uses the \texttt{renderDataTable()}
function from the
\emph{\href{https://CRAN.R-project.org/package=DT}{DT}} package (again,
this has a similar-but-not-identical function in
\emph{\href{https://CRAN.R-project.org/package=shiny}{shiny}}, so be
careful which one you import.) Some effort is involved in making sure
that the output table responds to the memorized parameter values of our
\texttt{GOTable} panel.

\begin{Shaded}
\begin{Highlighting}[]
\KeywordTok{setMethod}\NormalTok{(}\StringTok{".renderOutput"}\NormalTok{, }\StringTok{"GOTable"}\NormalTok{, }\ControlFlowTok{function}\NormalTok{(x, se, ..., output, pObjects, rObjects) \{}
    \KeywordTok{callNextMethod}\NormalTok{()}

\NormalTok{    panel_name <-}\StringTok{ }\KeywordTok{.getEncodedName}\NormalTok{(x)}
\NormalTok{    output[[panel_name]] <-}\StringTok{ }\NormalTok{DT}\OperatorTok{::}\KeywordTok{renderDataTable}\NormalTok{(\{}
        \KeywordTok{.trackUpdate}\NormalTok{(panel_name, rObjects)}
\NormalTok{        param_choices <-}\StringTok{ }\NormalTok{pObjects}\OperatorTok{$}\NormalTok{memory[[panel_name]]}

\NormalTok{        t.out <-}\StringTok{ }\KeywordTok{.retrieveOutput}\NormalTok{(panel_name, se, pObjects, rObjects)}
\NormalTok{        full_tab <-}\StringTok{ }\NormalTok{t.out}\OperatorTok{$}\NormalTok{contents}\OperatorTok{$}\NormalTok{table}
\NormalTok{        pObjects}\OperatorTok{$}\NormalTok{varname[[panel_name]] <-}\StringTok{ "tab"}

\NormalTok{        chosen <-}\StringTok{ }\NormalTok{param_choices[[}\StringTok{"Selected"}\NormalTok{]]}
\NormalTok{        search <-}\StringTok{ }\NormalTok{param_choices[[}\StringTok{"Search"}\NormalTok{]]}
\NormalTok{        search_col <-}\StringTok{ }\NormalTok{param_choices[[}\StringTok{"SearchColumns"}\NormalTok{]]}
\NormalTok{        search_col <-}\StringTok{ }\KeywordTok{lapply}\NormalTok{(search_col, }\DataTypeTok{FUN=}\ControlFlowTok{function}\NormalTok{(x) \{ }\KeywordTok{list}\NormalTok{(}\DataTypeTok{search=}\NormalTok{x) \})}

        \CommentTok{# If the existing row in memory doesn't exist in the current table, we}
        \CommentTok{# don't initialize it with any selection.}
\NormalTok{        idx <-}\StringTok{ }\KeywordTok{which}\NormalTok{(}\KeywordTok{rownames}\NormalTok{(full_tab)}\OperatorTok{==}\NormalTok{chosen)[}\DecValTok{1}\NormalTok{]}
        \ControlFlowTok{if}\NormalTok{ (}\OperatorTok{!}\KeywordTok{is.na}\NormalTok{(idx)) \{}
\NormalTok{            selection <-}\StringTok{ }\KeywordTok{list}\NormalTok{(}\DataTypeTok{mode=}\StringTok{"single"}\NormalTok{, }\DataTypeTok{selected=}\NormalTok{idx)}
\NormalTok{        \} }\ControlFlowTok{else}\NormalTok{ \{}
\NormalTok{            selection <-}\StringTok{ "single"}
\NormalTok{        \}}

\NormalTok{        DT}\OperatorTok{::}\KeywordTok{datatable}\NormalTok{(}
\NormalTok{            full_tab, }\DataTypeTok{filter=}\StringTok{"top"}\NormalTok{, }\DataTypeTok{rownames=}\OtherTok{TRUE}\NormalTok{,}
            \DataTypeTok{options=}\KeywordTok{list}\NormalTok{(}
                \DataTypeTok{search=}\KeywordTok{list}\NormalTok{(}\DataTypeTok{search=}\NormalTok{search, }\DataTypeTok{smart=}\OtherTok{FALSE}\NormalTok{, }\DataTypeTok{regex=}\OtherTok{TRUE}\NormalTok{, }\DataTypeTok{caseInsensitive=}\OtherTok{FALSE}\NormalTok{),}
                \DataTypeTok{searchCols=}\KeywordTok{c}\NormalTok{(}\KeywordTok{list}\NormalTok{(}\OtherTok{NULL}\NormalTok{), search_col), }\CommentTok{# row names are the first column!}
                \DataTypeTok{scrollX=}\OtherTok{TRUE}\NormalTok{),}
            \DataTypeTok{selection=}\NormalTok{selection}
\NormalTok{        )}
\NormalTok{    \})}
\NormalTok{\})}
\end{Highlighting}
\end{Shaded}

\section{Handling selections}\label{handling-selections}

Now for the most important bit - configuring the \texttt{GOTable} to
transmit a multiple row selection to other panels. This is achieved by
specializing a series of \texttt{.multiSelection*()} methods. The first
is the \texttt{.multiSelectionDimension()}, which controls the dimension
being transmitted:

\begin{Shaded}
\begin{Highlighting}[]
\KeywordTok{setMethod}\NormalTok{(}\StringTok{".multiSelectionDimension"}\NormalTok{, }\StringTok{"GOTable"}\NormalTok{, }\ControlFlowTok{function}\NormalTok{(x) }\StringTok{"row"}\NormalTok{)}
\end{Highlighting}
\end{Shaded}

The next most important method is the
\texttt{.multiSelectionCommands()}, which tells \texttt{iSEE()} how to
create the multiple row selection from the selected \texttt{DataTable}
row. It is expected to return a vector of commands that, when evaluated,
creates a character vector of row names for transmission. This has an
option (\texttt{index}) to differentiate between active and saved
selections, though the latter case is not relevant to our
\texttt{GOTable} so we will simply ignore it. We also need to protect
against cases where the requested GO term is not found, upon which we
simply return an empty character vector.

\begin{Shaded}
\begin{Highlighting}[]
\KeywordTok{setMethod}\NormalTok{(}\StringTok{".multiSelectionCommands"}\NormalTok{, }\StringTok{"GOTable"}\NormalTok{, }\ControlFlowTok{function}\NormalTok{(x, index) \{}
\NormalTok{    orgdb <-}\StringTok{ }\NormalTok{x[[}\StringTok{"Organism"}\NormalTok{]]}
\NormalTok{    type <-}\StringTok{ }\NormalTok{x[[}\StringTok{"IDType"}\NormalTok{]]}
    \KeywordTok{c}\NormalTok{(}
        \KeywordTok{sprintf}\NormalTok{(}\StringTok{"require(%s);"}\NormalTok{, orgdb),}
        \KeywordTok{sprintf}\NormalTok{(}\StringTok{"selected <- tryCatch(select(%s, keys=%s, keytype='GO', }
\StringTok{    column=%s)$SYMBOL, error=function(e) character(0));"}\NormalTok{, }
\NormalTok{            orgdb, }\KeywordTok{deparse}\NormalTok{(x[[}\StringTok{"Selected"}\NormalTok{]]), }\KeywordTok{deparse}\NormalTok{(type)),}
        \StringTok{"selected <- intersect(selected, rownames(se));"}
\NormalTok{    )}
\NormalTok{\})}
\end{Highlighting}
\end{Shaded}

We also define some generics to indicate whether a \texttt{DataTable}
row is currently selected, and how to delete that selection. For the
latter, we replace the selected row with an empty string to indicate
that no selection has been made, consistent with the actions of our
observer in \texttt{.createObservers()}.

\begin{Shaded}
\begin{Highlighting}[]
\KeywordTok{setMethod}\NormalTok{(}\StringTok{".multiSelectionActive"}\NormalTok{, }\StringTok{"GOTable"}\NormalTok{, }\ControlFlowTok{function}\NormalTok{(x) \{}
    \ControlFlowTok{if}\NormalTok{ (x[[}\StringTok{"Selected"}\NormalTok{]]}\OperatorTok{!=}\StringTok{""}\NormalTok{) \{}
\NormalTok{        x[[}\StringTok{"Selected"}\NormalTok{]]}
\NormalTok{    \} }\ControlFlowTok{else}\NormalTok{ \{}
        \OtherTok{NULL}
\NormalTok{    \}}
\NormalTok{\})}

\KeywordTok{setMethod}\NormalTok{(}\StringTok{".multiSelectionClear"}\NormalTok{, }\StringTok{"GOTable"}\NormalTok{, }\ControlFlowTok{function}\NormalTok{(x) \{}
\NormalTok{    x[[}\StringTok{"Selected"}\NormalTok{]] <-}\StringTok{ ""}
\NormalTok{    x}
\NormalTok{\})}
\end{Highlighting}
\end{Shaded}

Finally, we define a method to determine the total number of available
genes. The default is to use the number of rows of the
\texttt{data.frame} used in the \texttt{datatable()} call, but that
would not be right for us as it represents the number of gene sets.
Instead, we use the availability information that we previously stored
in the \texttt{contents} during \texttt{.generateOutput()}.

\begin{Shaded}
\begin{Highlighting}[]
\KeywordTok{setMethod}\NormalTok{(}\StringTok{".multiSelectionAvailable"}\NormalTok{, }\StringTok{"GOTable"}\NormalTok{, }\ControlFlowTok{function}\NormalTok{(x, contents) \{}
\NormalTok{    contents}\OperatorTok{$}\NormalTok{available}
\NormalTok{\})}
\end{Highlighting}
\end{Shaded}

\section{In action}\label{in-action-3}

Let's put our new panel to the test using the \texttt{sce} object,
preprocessed in a \protect\hyperlink{developing}{previous chapter}.

Setting up the iSEE instance is as easy as:

\begin{Shaded}
\begin{Highlighting}[]
\NormalTok{got <-}\StringTok{ }\KeywordTok{new}\NormalTok{(}\StringTok{"GOTable"}\NormalTok{, }\DataTypeTok{PanelWidth=}\NormalTok{8L)}
\NormalTok{rst <-}\StringTok{ }\KeywordTok{RowDataTable}\NormalTok{(}\DataTypeTok{RowSelectionSource=}\StringTok{"GOTable1"}\NormalTok{)}
\NormalTok{app <-}\StringTok{ }\KeywordTok{iSEE}\NormalTok{(sce, }\DataTypeTok{initial=}\KeywordTok{list}\NormalTok{(got, rst))}
\end{Highlighting}
\end{Shaded}

Clicking on any row in the \texttt{GOTable} will subset
\texttt{RowTable1} to only those genes in the corresponding GO term.

\bibliography{book.bib,packages.bib}

\end{document}
